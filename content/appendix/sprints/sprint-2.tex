\subsection*{Sprint 2}

\subsubsection*{Zusammenfassung}

\begin{table}[H]
	\centering
	\begin{tabular}{ll}
		\toprule
		\multicolumn{2}{c}{\textbf{Sprint 2} \textit{(2 Wochen, Construction)}}\\
		\midrule
		\textbf{Periode} & 04.04.2016\textendash 17.04.2016\\
		\textbf{Stunden Soll} & \SI{80}{\hour}\\
		\textbf{Stunden Ist} & \SI{81.5}{\hour}\\
		\bottomrule
	\end{tabular}	
\end{table}


\subsubsection*{Ziele}
\begin{itemize}
	\item MS5: Anstehende Updates anzeigen
	\item CC: Setup-Guide erstellen
	\item CC: Unittest erstellen
	\item CC: Multithreading abklären
	\item CC: Trennen von Config- und Testdaten
	\item A: Daemonize und Schedule implementieren
	\item Zwischenpräsentation vorbereiten
\end{itemize}


\subsubsection*{Erledigt}
MS5 wurde erreicht. Die Zwischenpräsentation wurde vorbereitet und gehalten. Im Control Center wurden die gemeldeten Updates bereits korrekt angezeigt. Basisdaten für den Gebrauch des Control Centers wurden eingerichtet und als Rake-Task erstellt, so dass sie bei der Installation jeweils mit aufgesetzt werden können.
Unit-Tests wurden erstellt.

\subsubsection*{Probleme}
An der Zwischenpräsentation wurde festgestellt, dass Redundanzen in der Datenüberertragung zu einer unnötigen Last führen. Ein Task zur Behebung wurde für Sprint 3 geplant.