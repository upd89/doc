\subsection*{Sprint 4}

\subsubsection*{Zusammenfassung}

\begin{table}[H]
	\centering
	\begin{tabular}{ll}
		\toprule
		\multicolumn{2}{c}{\textbf{Sprint 4} \textit{(2 Wochen, Construction)}}\\
		\midrule
		\textbf{Periode} & 02.05.2016\textendash 15.05.2016\\
		\textbf{Stunden Soll} & \SI{80}{\hour}\\
		\textbf{Stunden Ist} & \SI{82.5}{\hour}\\
		\bottomrule
	\end{tabular}	
\end{table}


\subsubsection*{Ziele}
\begin{itemize}
	\item MS7: UC02: Registration Agent an Control Center
	\item MS7: UC07: Berechtigungen verwalten
	\item MS7: UC09: User verwalten
	\item CC und A: TLS implementiert
	\item CC und A: Redundanzen bei API Calls verringern
	\item CC: User Authentifizierung, Rechtesystem implementieren
	\item CC: Filter und Ordering in JS implementieren
	\item A: Logfile an Server übermitteln
\end{itemize}


\subsubsection*{Erledigt}
MS7 wurde erreicht. Die Agenten konnten sich beim Control Center registrieren und die Systeme konnten dort verwaltet werden. Usermanagement und Login-Funktionalität wurden implementiert sowie Filter- und Sortiermöglichkeit eingebaut. Das Rechtesystem wurde ebenfalls implementiert.

\subsubsection*{Probleme}
Bei Löschoperationen weigerte sich Rails die Löschung vorzunehmen, falls Referenzen auf das zu löschende Objekt existieren. Das Vorgehen diesbezüglich wurde deshalb mit \gls{nine} besprochen.

Das Gem für die Filter (Filteriffic) unterstützt leider nur einen Active-Record-Eintrag pro View, weswegen die kombinierte Paket- und System-View noch nicht fertiggestellt wurde.