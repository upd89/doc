\subsection*{Sprint 1}

\subsubsection*{Zusammenfassung}

\begin{table}[H]
	\centering
	\begin{tabular}{ll}
		\toprule
		\multicolumn{2}{c}{\textbf{Sprint 1} \textit{(2 Wochen, Construction)}}\\
		\midrule
		\textbf{Periode} & 21.03.2016\textendash 03.04.2016\\
		\textbf{Stunden Soll} & \SI{60}{\hour}\\
		\textbf{Stunden Ist} & \SI{72}{\hour}\\
		\bottomrule
	\end{tabular}	
\end{table}


\subsubsection*{Ziele}
\begin{itemize}
	\item MS4: UC03: Anstehende Updates melden
	\item CC: API Umsetzen
	\item CC: Datenbank aufsetzen
	\item A: apt-Infos auslesen
\end{itemize}


\subsubsection*{Erledigt}
MS4 wurde erreicht. Domänenmodell wurde überarbeitet nach Input vom Betreuer. Der Agent konnte auf manuellen Input hin über apt die anstehenden Updates auslesen und an die umgesetzte erste Version der API auf dem Control Center melden. Die Datenbank wurde mit dem Aufsetzen des Rails-Projektes bereits installiert und war einsatzbereit.

\subsubsection*{Probleme}
In der Entwicklungsumgebung konnte nur ein Agent gleichzeitig bedient werden. Ein Task zur Umstellung auf Multithreading wurde deshalb für Sprint 2 geplant.
