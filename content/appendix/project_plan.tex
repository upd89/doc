\chapter{Projektplan}

Die Bachelorarbeit findet im Frühlingssemester 2016 statt und dauert 17 Wochen. Das Startdatum fällt auf den Semesterstart vom 22.02.2016, das Enddatum ist der 17.05.2016.

Für jede Person wird mit einem Zeitbudget von 12 x 30 Stunden gerechnet. Insgesamt können im Projekt mit 720 Arbeitsstunden geplant werden. Dies beinhaltet sämtliche im Rahmen der Bachelorarbeit vorgenommenen Arbeiten, Kommunikation sowie Dokumentation.

Das gesamte Projekt wird im Stil von RUP geführt und dementsprechend in die vier Phasen Inception, Elaboration, Construction und Transition aufgeteilt. In der Construction-Phase wird nach agiler Vorgehensweise gearbeitet, so dass der Auftraggeber die Schwerpunkte setzen und priorisieren kann.

Es ist vorgesehen, dass die Inception-Phase eine Woche dauert. Danach findet die dreiwöchige Elaboration-Phase statt, in welcher die Konstruktionsphase geplant und der Grossteil der Anforderungen geklärt wird.
In der zehnwöchigen Construction-Phase findet die eigentliche Umsetzung auf Grund der vorher erarbeiteten Rahmenbedingungen und Anforderungen statt. Im Anschluss folgt die Transition- oder Übergabe-Phase über drei Wochen, wo auch die Präsentation vorbereitet und die Dokumentation abgeschlossen wird.

\section{Phasen/Iterationen}

\begin{table}[H]
    \centering
    \caption{Phasen/Iterationen}
    \label{phases}
    \begin{tabular}{| l | l | l | l |}
        \toprule
        Woche & Phase/Iteration                 & Meilenstein & Tasks                                                                      \\
        \midrule
        1     & Inception                       & 1           & Projektauftrag schreiben, Infrastruktur bestellen                          \\
        2     & Elaboration 1                   & 2           & Projektplan schreiben, Arbeitspakete bestimmen, Setup Projektinfrastruktur \\
        3     & Elaboration 2                   &             & Risikoanalyse, Domainanalyse entwerfen, UseCase-Analyse                    \\
        4     & Elaboration 3                   & 3           & UI und API entwerfen, Toolchain und Testumgebung bereitstellen             \\
        5     & \multirow{2}{*}{Construction 1} &             & \multirow{2}{*}{Sprint 1}                                                  \\
        6     &                                 & 4           &                                                                            \\
        7     & \multirow{2}{*}{Construction 2} &             & \multirow{2}{*}{Sprint 2}                                                  \\
        8     &                                 & 5           &                                                                            \\
        9     & \multirow{2}{*}{Construction 3} &             & \multirow{2}{*}{Sprint 3}                                                  \\
        10    &                                 & 6           &                                                                            \\
        11    & \multirow{2}{*}{Construction 4} &             & \multirow{2}{*}{Sprint 4}                                                  \\
        12    &                                 & 7           &                                                                            \\
        13    & \multirow{2}{*}{Construction 5} &             & Sprint 5                                                                   \\
        14    &                                 & 8           & Cleanup                                                                    \\
        15    & Transition 1                    &             & Produkte-Dokumentation schreiben                                           \\
        16    & Transition 2                    & 9           & Abschluss Dokumentation                                                    \\
        17    & Transition 3                    &             & Vorbereitung Präsentation                                                  \\
        \bottomrule
    \end{tabular}
\end{table}


\section{Meilensteine}

\begin{itemize}
    \item MS1: 29.02.2016, Abgabe Aufgabenstellung
    \item MS2: Entwurf Projektplan
    \item MS3: End of Elaboration
    \item MS4: UC03 “Ausstehende Updates melden” umgesetzt (apt-Infos auslesen, übermitteln an CC, speichern)
    \item MS5: Statusupdates werden angezeigt, Agent ist ein Daemon inkl. Job-Queue und Schedule
    \item MS6: UC05 “Updates auslösen” und UC04 “Update durchführen” umgesetzt
	(Updates können freigegeben und an den Agent geschickt werden)
    \item MS7: UC02 “Bei Control-Center registrieren”, UC07 “Berechtigungen verwalten” und UC09 “User verwalten” umgesetzt, sowie TLS implementiert
    \item MS8: UC06 “Reports einsehen”, UC08 “System-Gruppen verwalten”, UC10 “Regeln verwalten” umgesetzt
    \item MS9: Juni 2016: Abgabe Finale Version
\end{itemize}