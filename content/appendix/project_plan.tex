\chapter{Projektplan}
\label{appendix:project_plan}

Die Bachelorarbeit findet im Frühlingssemester 2016 statt und dauert 17 Wochen. Das Startdatum fällt auf den Semesterstart vom 22.02.2016, das Enddatum ist der 17.05.2016.

Für jede Person wird mit einem Zeitbudget von 12 x 30 Stunden gerechnet. Insgesamt können im Projekt mit 720 Arbeitsstunden geplant werden. Dies beinhaltet sämtliche im Rahmen der Bachelorarbeit vorgenommenen Arbeiten, Kommunikation sowie Dokumentation.

Das gesamte Projekt wird im Stil von RUP geführt und dementsprechend in die vier Phasen Inception, Elaboration, Construction und Transition aufgeteilt. In der Construction-Phase wird nach agiler Vorgehensweise gearbeitet, so dass der Auftraggeber die Schwerpunkte setzen und priorisieren kann.

Es ist vorgesehen, dass die Inception-Phase eine Woche dauert. Danach findet die dreiwöchige Elaboration-Phase statt, in welcher die Konstruktionsphase geplant und der Grossteil der Anforderungen geklärt wird.
In der zehnwöchigen Construction-Phase findet die eigentliche Umsetzung auf Grund der vorher erarbeiteten Rahmenbedingungen und Anforderungen statt. Im Anschluss folgt die Transition- oder Übergabe-Phase über drei Wochen, wo auch die Präsentation vorbereitet und die Dokumentation abgeschlossen wird.

\section*{Phasen/Iterationen}

\begin{table}[H]
    \centering
    \caption{Phasen/Iterationen}
    \label{phases}
    \begin{tabular}{| l | l | l | l |}
        \toprule
        Woche & Phase/Iteration                 & Meilenstein & Tasks                                                     \\
        \midrule
        1     & Inception                       & MS1         & Projektauftrag schreiben, Infrastruktur bestellen       \\
        2     & Elaboration 1                   & MS2         & Projektplan, Arbeitspakete, Setup Projektinfrastruktur  \\
        3     & Elaboration 2                   &             & Risikoanalyse, Domainanalyse entwerfen, UseCase-Analyse \\
        4     & Elaboration 3                   & MS3         & UI und API entwerfen, Toolchain und Testumgebung        \\
        5     & \multirow{2}{*}{Construction 1} &             & \multirow{2}{*}{Sprint 1}                               \\
        6     &                                 & MS4         &                                                         \\
        7     & \multirow{2}{*}{Construction 2} &             & \multirow{2}{*}{Sprint 2}                               \\
        8     &                                 & MS5         &                                                         \\
        9     & \multirow{2}{*}{Construction 3} &             & \multirow{2}{*}{Sprint 3}                               \\
        10    &                                 & MS6         &                                                         \\
        11    & \multirow{2}{*}{Construction 4} &             & \multirow{2}{*}{Sprint 4}                               \\
        12    &                                 & MS7         &                                                         \\
        13    & \multirow{2}{*}{Construction 5} &             & Sprint 5                                                \\
        14    &                                 & MS8         & Cleanup                                                 \\
        15    & Transition 1                    &             & Produkte-Dokumentation schreiben                        \\
        16    & Transition 2                    & MS9         & Abschluss Dokumentation                                 \\
        17    & Transition 3                    &             & Vorbereitung Präsentation                               \\
        \bottomrule
    \end{tabular}
\end{table}


\section*{Meilensteine}

\begin{itemize}
    \item MS1: Abgabe Aufgabenstellung (29.02.2016)
    \item MS2: Entwurf Projektplan
    \item MS3: End of Elaboration
    \item MS4: \nameref{sec:uc_03} umgesetzt, apt-Infos auslesen, übermitteln an CC, speichern
    \item MS5: Statusupdates werden angezeigt, Agent ist ein Daemon inkl. Job-Queue und Schedule
    \item MS6: \nameref{sec:uc_05} und \nameref{sec:uc_04} umgesetzt, Updates können freigegeben und an den Agent geschickt werden
    \item MS7: \nameref{sec:uc_02}, \nameref{sec:uc_07} und \nameref{sec:uc_09} umgesetzt, sowie TLS implementiert
    \item MS8: \nameref{sec:uc_06}, \nameref{sec:uc_08}, \nameref{sec:uc_10} umgesetzt
    \item MS9: Abgabe Finale Version (17.05.2016)
\end{itemize}