\section*{Ueli Bosshard}

Diese Bachelorarbeit lag mir aus verschiedenen Gründen ganz besonders am Herzen. Da der Auftrag von einem Industriepartner vergeben wurde, wusste ich, dass es nicht ein Schulprojekt ist, das in der Versenkung verschwindet, sobald es fertig ist. Weiter wird die Arbeit als Open-Source-Projekt veröffentlicht, womit ich das Projekt auch als Referenz in meinem Lebenslauf verwenden kann. Ausserdem hat die Arbeit auch einige Komponenten aus der Systemtechnik miteinbezogen, bei denen ich auf meine Erfahrung als Systemtechniker zurückgreifen konnte.

Als das Schwierigste an der Arbeit empfand ich die Planung zu Beginn der Arbeit. Es fiel uns nicht weiter schwer das Projekt in grobe Teilaufgaben zu unterteilen, aber bei der Detailplanung gab es immer wieder Herausforderungen, die bei der Grobplanung nicht auffielen. So gab es Momente in der Entwicklung, bei denen wir die ursprüngliche Planung in Frage stellen mussten. Glücklicherweise konnten wir uns an die Grobplanung halten, mussten aber immer wieder kleinere Korrekturen vornehmen.

Das wir alle geforderten Anforderungen in der vorgegeben Zeit abschliessen konnten, ist nach meiner Meinung einerseits dem Umstand zu verdanken, dass wir bereits sehr früh eine Teilaufgabe (Use Case 10) streichen konnten. Andererseits haben sich aber auch unsere Stärken sehr gut ergänzt, womit wir nicht nur eine funktionierende Lösung, sondern dank Philipp auch ein professionelles Webinterface vorweisen können.

Ein Risiko der Arbeit bestand darin, dass wir beide noch nicht viel mit Ruby on Rails gearbeitet haben. Auch wenn wir uns erst in das Framework einarbeiten mussten, hat es sich am Ende doch gelohnt. Ruby on Rails hat uns sehr viel Arbeit abgenommen und ich würde es für ein ähnliches Projekt sofort wieder verwenden. So wurde aus dieser Arbeit ein Produkt, auf das ich stolz bin.

Zu guter Letzt möchte ich mich bei einigen Personen für die Unterstützung bei dieser Bachelorarbeit bedanken. Bei Sam und Roland von nine.ch für das Ausschreiben und die Unterstützung der Arbeit, bei Prof. Dr. Mehta für die Betreuung, bei allen Lektoren für die hilfreichen Rückmeldungen und natürlich ganz besonders bei Philipp Christen für die vielen Stunden bei denen wir gemeinsam Lösungen und Formulierungen gesucht haben.