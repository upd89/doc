\section*{Ueli Bosshard}

Diese Bachelorarbeit lag aus verschiedenen Gründen ganz besonders am Herzen. Da der Auftrag von einem Industriepartner vergeben wurde, wusste ich, das es nicht ein Schulprojekt ist, das in der Versenkung verschwindet, sobald es fertig ist. Weiter wird die Arbeit als Open-Source-Projekt veröffentlicht, womit ich das Projekt auch als Referenz in meinem Lebenslauf verwenden kann. Ausserdem hat die Arbeit auch einige Komponenten aus der Systemtechnik miteinbezogen, bei denen ich auf meine Erfahrung als Systemtechinker zurückgreifen konnte. 

Als das Schwierigste an der Arbeit empfand ich die Planung zu Beginn der Arbeit. Es fiel uns nicht weiter schwer das Projekt in grobe Teilaufgaben zu unterteilen, aber bei der Detailplanung gab es immer wieder Herausforderungen, die bei der Grobplanung nicht auffielen. So gab es Momente in der Entwicklung, bei denen wir die ursprüngliche Planung in Frage stellen mussten. Glücklicherweise konnten wir uns an die Grobplanung halten, mussten aber immer wieder kleinere Korrekturen vornehmen.

Das wir alle geforderten Anforderungen in der vorgegeben Zeit abschliessen konnten, ist nach meiner Meinung dem Umstand zu verdanken, dass wir bereits sehr früh eine Teilaufgabe (Use Case 10) streichen konnten.