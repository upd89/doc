\section*{Philipp Christen}

Während der Bachelorarbeit arbeitete ich hauptsächlich am Frontend, was durch meine bisherige Erfahrung und berufliche Tätigkeit auch kein Problem war.

Besonders das Planen und Erstellen der Views mit Mockups sowie das Kennenlernen von Rails war spannend. Da wir beide noch keine Kenntnisse von Ruby hatten, dauerte es eine Weile bis wir einigermassen vertraut damit waren. Leider mussten wir im Verlauf immer wieder feststellen, dass unsere Lösungen einfacher, effizienter und eleganter umgesetzt hätten werden können. Rückblickend und mit den jetzigen Kenntnissen hätten wir sicher andere Entscheidungen getroffen. Beispielsweise müssten Fragen zur Concurrency und Skalierbarkeit früher untersucht werden

Die Zusammenarbeit mit nine.ch war angenehm und hilfreich. Wir wurden sowohl mit Infrastruktur als auch mit Ratschlägen zu Rails unterstützt und konnten so einige Entscheidungen einfacher fällen.

Gerne hätte ich noch mehr Zeit in die Usability und Verbesserung des Frontends investiert, wo es noch offene Punkte gibt. Insgesamt bin ich aber zufrieden mit dem Resultat: Wir haben eine funktionierende Applikation entwickelt, welche ein konkretes Problem löst, kostenlos und als Open Source angeboten wird. Nebenbei lernte ich einiges über Applikationssicherheit und Serveradministration, wobei Ueli mir aufgrund seiner Erfahrungen vieles erklären konnte. Auch im Backend-Bereich war die Arbeit für mich lehrreich, da ich in den bisherigen Arbeiten und im Berufsalltag sehr selten mit Datenbanken und Concurrency-Problemen zu kämpfen hatte.

Ich hoffe, dass die Applikation auch ausserhalb von nine.ch einige Anwender findet und vielleicht sogar weiterentwickelt wird.

Abschliessend möchte ich mich bei Ueli Bosshard für die gute Zusammenarbeit sowie bei Sarah Schoch für die Unterstützung und das Gegenlesen der Arbeit bedanken. H.B. und J.P. haben ebenso zum Gelingen der Arbeit beigetragen.
