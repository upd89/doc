\section*{Philipp Christen}

Während der Bachelorarbeit arbeitete ich hauptsächlich am Control Center, wobei mir das Umsetzen des Frontends durch meine bisherige Erfahrung und berufliche Tätigkeit keine Problem verursachte.

Besonders das Planen und Erstellen der Views mit Mockups, sowie das Kennenlernen von Rails, war für mich spannend. Da wir beide noch fast keine Kenntnisse von Ruby hatten, dauerte es eine Weile bis wir damit vertraut waren. Leider mussten wir im Verlauf der Arbeit immer wieder feststellen, dass unsere Lösungen einfacher, effizienter und eleganter umsetzbar gewesen wären. Rückblickend und mit den jetzigen Kenntnissen, würden wir sicher einige Hindernisse anders angehen. Beispielsweise müssten Fragen zur Concurrency und Skalierbarkeit früher untersucht werden.

Die Zusammenarbeit mit nine.ch war angenehm und hilfreich. Wir wurden sowohl mit Infrastruktur als auch mit Ratschlägen zu Rails unterstützt und konnten so einige Entscheidungen einfacher fällen.

Gerne hätte ich noch mehr Zeit in die Usability und Verbesserung des Frontends investiert, wo es noch offene Punkte gibt. Insgesamt bin ich aber zufrieden mit dem Resultat: Wir haben eine funktionierende Applikation entwickelt, welche ein konkretes Problem löst, kostenlos und als Open Source angeboten wird. Nebenbei lernte ich einiges über Applikationssicherheit und Serveradministration, wobei Ueli mir aufgrund seiner Erfahrungen vieles erklären konnte. Auch im Backend-Bereich war die Arbeit für mich lehrreich, da ich in den bisherigen Arbeiten und im Berufsalltag nur selten mit Datenbanken und Concurrency-Problemen zu kämpfen hatte.

Ich hoffe, dass die Applikation auch ausserhalb von nine.ch einige Anwender findet und weiterentwickelt wird.

Abschliessend möchte ich mich bei Ueli Bosshard für die gute Zusammenarbeit sowie bei Sarah Schoch für die Unterstützung und das Gegenlesen der Arbeit bedanken. H.B. und J.P. haben ebenso zum Gelingen der Arbeit beigetragen. Prof. Dr. Mehta danke ich für die Betreuung und Hilfestellungen.
