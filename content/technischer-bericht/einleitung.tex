\begin{comment}
Einführung in die Problem- und Aufgabenstellung. Übersicht über die übrigen Teile der Abgabe.

Diese Einleitung soll für den Ingenieur irgendeiner Fachrichtung verständlich sein. Sie stellt die Aufgabe in einen grösseren Zusammenhang und liefert eine genaue Beschreibung der Problemstellung. Allfällige Vorarbeiten oder ähnlich gelagerte Arbeiten werden diskutiert.
\end{comment}

\chapter{Einleitung und Übersicht}
\xxx[]

Im Auftrag des Industriepartners wurde eine moderne Lösung für das zentrale Updatemanagement erarbeitet. Was zuvor per Shellscript erledigt wurde ist nun übersichtlich über einen Browser-Basiertes Webinterface möglich. Der Administrator profitiert dabei von den aktellen Statusinformationen, die ihn bei seinen Überlegungen unterstützen.

Dieser Bericht widerspiegelt das Vorgehen der Bachelorarbeit und geht in die einzelnen Phasen der Arbeit ein.

Die nachfolgenden Kapitel beschäftigen sich mit der Feststellung der \textbf{Anforderungen} und der \textbf{Analyse} aus welcher die Formulierung der Use Cases, Abuse Cases und des Domänenmodells resultiert. Im Abschluss der Analyse wird mit den Umsetzungen der Konkurrenz verglichen, um zu zeigen, wie sich diese Lösung von anderen Lösungen unterscheidet.

Anschliessend wird in der \textbf{Umsetzung} auf die Architektur eingegangen. Besonders wichtig bei der Umsetzung waren Überlegungen zur sicheren Kommunikation zwischen den einzelnen Komponenten. Ein weiterer Schwerpunkt liegt auf der Gestaltung der Benutzeroberfläche, da dies der Hauptgrund für den Wechsel vom Shellscript zum Webinterface darstellt.

Das \textbf{Ergebnis} wird in einem eigenen Kapitel gezeigt und bewertet. Schliesslich zeigt der \textbf{Ausblick} mögliche Erweiterungsschritte.

Im letzten Kaptiel wird auf das \textbf{Projektmanagement} eingegangen.
