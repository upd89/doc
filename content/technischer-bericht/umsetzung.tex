\begin{comment}
(Implementierung) Architektur und Design beschrieben: Mit begründeten Architekturentscheidungen, mit Diskussion, wie Qualitätsattribute sichergestellt wurden (welche Qualität wurde erreicht?), mit Dokumentation, welche Experimente/Tests durchgeführt wurden und welche Lösungsoptionen aufgrund der Ergebnisse dieser Experimente/Tests
verworfen wurden (was ging schief?)
\end{comment}

\chapter{Umsetzung} \label{sec:implementation}

Dieses Kapitel beschreibt die Umsetzung anhand der Architektur und legt einen Schwerpunkt auf die Themen Sicherheit und Design.

\subimport{}{architecture.tex}

\clearpage
\subimport{}{security.tex}

\clearpage
\subimport{}{design.tex}

\clearpage
\subimport{}{testing.tex}

\section{Dokumentation} \label{sec:implementation:documentation}

\begin{decision}{Dokumentationssprache}
Es wurde entschieden, dass die Projektdokumentation - etwa dieses Dokument - auf Deutsch verfasst wird, die Software selbst sowie Anleitungen dazu auf Englisch veröffentlich werden. Dies aus dem Grund, da Englisch die meistbenutzte Sprache im Internet ist \cite{websitelanguages} und aus eigener Erfahrung die meisten Open-Source-Projekte mindestens auf Englisch dokumentiert sind.
\end{decision}

\begin{decision}{Dokumentations-Technologie}
Die Dokumentation wurde in Latex verfasst. Dies wurde primär aufgrund der vorherigen Erfahrung entschieden. Beide Teammitglieder hatten bereits Erfahrungen mit Latex gesammelt und vorherige Arbeiten damit erstellt.
\end{decision}