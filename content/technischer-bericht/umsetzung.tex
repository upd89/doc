\chapter{Umsetzung}

\xxx[(Implementierung) Architektur und Design beschrieben: Mit begründeten Architekturentscheidungen, mit Diskussion, wie Qualitätsattribute sichergestellt wurden (welche Qualität wurde erreicht?), mit Dokumentation, welche Experimente/Tests durchgeführt wurden und welche Lösungsoptionen aufgrund der Ergebnisse dieser Experimente/Tests
verworfen wurden (was ging schief?)]

\section{Architektur}

\xxx[Mit begründeten Architekturentscheidungen, Mit Diskussion, wie Q-Attribute sichergestellt wurden (welche Qualität wurde erreicht?), Mit Dokumentation, welche Tests durchgeführt wurden, welche Lösungsoptionen wurde aufgrund von Tests/Experimenten verworfen.]

\subsection*{Control Center}

Das Control Center wurde mit Ruby on Rails umgesetzt. Rails ist ein \gls{mvc}-Framework, wodurch grundsätzliche Entscheidungen zur Architektur bereits vorweggenommen wurden.

Das \gls{mvc}-Pattern teilt Applikationen in drei Teile, welche jeweils verschiedene Aufgaben und Verantwortungen haben. Ziel ist, dass die Wiederverwendbarkeit und Trennbarkeit der Komponenten gewährleistet wird.

Ruby on Rails bildet das Pattern bereits in der Dokumentenstruktur ab. Im /app befinden sich Ordner für Controllers, Models und Views.

\begin{figure}[H]
	\centering
	\includegraphics[width=0.5\linewidth]{files/mvc_structure}
	\caption{MVC-Pattern}
	\label{fig:tec:mvc}
\end{figure}

\subsection*{Agent}

\xxx

\clearpage
\subimport{}{security.tex}

\clearpage
\subimport{}{design.tex}

\clearpage
\section{Testing}

\xxx[Unit-, UI-, Akzeptanztests]
