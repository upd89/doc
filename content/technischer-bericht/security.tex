\section{Sicherheit}
\label{sec:security}

In der Analyse der Anforderungen wurde festgestellt, dass auf die Sicherheit besonderes Augenmerk gelegt werden muss. Insbesondere die festgestellten Abuse Cases (siehe \ref{sec:abuse_cases}) mussten berücksichtigt werden.

\subsection*{Übersicht}

Im Fokus stand eine vertrauenswürdige Verbindung zwischen allen Teilnehmern des Systems. Namentlich sind das die Agents, das Control Center, sowie der Administrator, der per Browser auf die Administrationsoberfläche zugreift.

\begin{figure}
  \centering
    \includegraphics[width=0.75\textwidth]{files/upd89-security-overview}
  \caption{Übersicht Verbindungssicherheit}
  \label{fig:sec-overview}
\end{figure}

Jeder Teilnehmer muss sich sicher sein, dass er mit dem korrekten Gegenüber kommuniziert.

\begin{enumerate}[label=\color{orange}\theenumi]
    \item Der \gls{agent} überprüft das Zertifikat des \glspl{controlcenter}, damit er weiss, dass er den Auftrag ausführen darf, den er auf diesem Weg bekommt. Ausserdem stellt er sicher, dass er Statusinformationen an das korrekte Gegenüber abliefert.
    \item Das \gls{controlcenter} seinerseits stellt ebenfalls mit dem Zertifikat des Agents fest, ob er dem Gegenüber vertrauen kann und die gelieferten Informationen für ihn relevant und korrekt sind.
    \item Der Administrator kann anhand des Zertifikats des Servers feststellen, ob er mit dem korrekten \gls{controlcenter} verbunden ist.
    \item Das \gls{controlcenter} vertraut dem Administrator anhand des eingegebenen Passworts.
\end{enumerate}

\subsection*{API Security}

Um die Kommunikation zwischen Agent und Control Center abzusichern, wird eine eigene \gls{ca} eingesetzt. Im Grunde kann die \gls{ca} mit Openssl \footnote{Openssl-Webseite: \purl{https://www.openssl.org/}} erstellt werden. Als Unterstützung wird easy-rsa \footnote{easy-rsa-Webseite: \purl{https://openvpn.net/easyrsa.html}} verwendet, um die benötigten Zertifikate mit wenig Aufwand zu generien.

\subsubsection*{Erzeugen der Zertifikate}
Das Konzept einer Zertifizierungsstelle (\gls{ca}) sieht wie folgt aus: Die \gls{ca} unterschreibt den eigenen öffentlichen Schlüssel (pub) und generiert daraus das öffenliche CA-Zertifikat (CA-crt). Anschliessend unterschreibt die \gls{ca} die öffentlichen Schlüssel der Teilnehmer, in unserem Fall das Control Center und die Agents.

\begin{figure}
  \centering
    \includegraphics[width=0.60\textwidth]{files/upd89-ca-signing}
  \caption{Erzeugen von Zertifikaten mittels Signieren}
  \label{fig:sec-signin}
\end{figure}

\clearpage
\subsubsection*{Sichere Verbindung}

Die Teilnehmer sind im Besitz des \gls{ca}-Zertifikats, welches als Vertrauensanker dient, können so die Zertifikate ihrer Kommunikationspartner überprüfen und so eine sichere Verbindung gewährleisten.

\begin{figure}
  \centering
    \includegraphics[width=0.70\textwidth]{files/upd89-secure-communication}
  \caption{Vertrauenswürdige Verbindung aufgrund Vertrauensanker}
  \label{fig:sec-connection}
\end{figure}
