\section{Design und Implementation}

\xxx

\subsection*{Mockups}

Als Diskussionsgrundlage und Vorlage für die Umsetzung wurden ca. 30 verschiedene \glspl{mockup} mit Balsamiq\footnote{\purl{https://balsamiq.com/products/mockups/}} umgesetzt.

\subsubsection*{Mockups}



\begin{figure}[H]
	\centering
	\includegraphics[width=\linewidth]{files/mockups/group_systems}
	\caption{Konzept Workflow-Feature}
	\label{fig:design:group_users_mockup}
\end{figure}



\begin{figure}[H]
	\centering
	\includegraphics[width=0.75\linewidth]{files/mockups/permission_1}
	\caption{Rechtesystem}
	\label{fig:design:permission_1}
\end{figure}

\begin{figure}[H]
	\centering
	\includegraphics[width=0.75\linewidth]{files/mockups/permission_2}
	\caption{Rechtesystem (Alternative)}
	\label{fig:design:permission_2}
\end{figure}

\begin{figure}[H]
	\centering
	\includegraphics[width=\linewidth]{files/mockups/dashboard}
	\caption{Konzept des Dashboards}
	\label{fig:design:dashboard_mockup}
\end{figure}

\begin{figure}[H]
	\centering
	\includegraphics[width=\linewidth]{files/mockups/combo_view}
	\caption{Haupt-Ansicht mit Paket- und System-Filter}
	\label{fig:design:combo_view_mockup}
\end{figure}

\subsection*{User Interface}

Die Oberflächen halten sich mehrheitlich an die Mockups aus der Planung, einige Details wurden aber aufgrund von Input durch die Industriepartner oder durch 'Hallway-Testing'\footnote{Test mit zufällig ausgewählten Personen, ob ein Feature in der Benutzeroberfläche wie gedacht funktioniert.} abgeändert.

\begin{figure}[H]
	\centering
	\includegraphics[width=\linewidth]{files/upd89-screenshot_dashboard}
	\caption{Umgesetztes Dashboard}
	\label{fig:design:dashboard}
\end{figure}

\subsubsection*{Farben}

Generell wurde das Interface eher schlicht gehalten. Grautöne sowie Rot, Gelb und Grün als Signalfarben (\ref{fig:design:messages}) lassen die Seiten übersichtlich und nicht überladen erscheinen.

\begin{figure}[H]
	\centering
	\includegraphics[width=0.5\linewidth]{files/messages}
	\caption{Beispiele der drei Hinweise 'Erfolg', 'Warnung', 'Error'}
	\label{fig:design:messages}
\end{figure}

Die selbst definierten Farben wurden in einer separaten \gls{scss}-Datei definiert, wodurch es für anwendungs- oder benutzerspezifische Anpassungen einfach abzuändern ist. Eine Übersicht über die wichtigsten Farben ist in Bild \ref{fig:design:colorcodes} zu sehen.

\begin{figure}[H]
	\centering
	\includegraphics[width=\linewidth]{files/colorcodes}
	\caption{Verwendete Farben mit Hex-Codes}
	\label{fig:design:colorcodes}
\end{figure}

Ein besonderes Augenmerk wurde auf die farbliche Kennzeichnung der Systeme und Pakete gelegt. Besonders in der kombinierten Ansicht, wo es Filtermöglichkeiten zu Systemen und zu Paketen gibt, ist es mit einer farblichen Unterscheidung einfacher zu sehen, was welche Entitätsgruppe betrifft. Diese Farben wurden auch ins Menu und ins Dashboard weitergezogen; in der ganzen Applikation sind Infos und Einstellungen, welche zu Systemen relevant sind, generell im 'System-Blau', solche für Pakete im 'Paket-Grün' (Beispiel im Bild \ref{fig:design:sys_pkg_colors}).

\xxx[ref zu combo-view?]

\begin{figure}[H]
	\centering
	\includegraphics[width=\linewidth]{files/colors_pkg_sys}
	\caption{Simple Farb-Hinweise: Blau für Systeme, Grünlich für Pakete}
	\label{fig:design:sys_pkg_colors}
\end{figure}

\subsubsection*{Building Blocks}

Um verschiedene grafische Elemente konsistent gleich aussehen zu lassen, wurde das HTML Kickstart-Framework von Joshua Gatcke/99lime.com\footnote{\purl{http://www.99lime.com/elements/}} verwendet. Dadurch waren gewisse Funktionalitäten wie etwa das Menu oder die Hinweise bereits vorhanden.


Für die Icons wurde FontAwesome\footnote{\purl{http://fontawesome.io/}} verwendet.
