\begin{comment}
Ergebnisse der Arbeit: was wurde erreicht, was wurde nicht erreicht, Ursachen.
Dieser Abschnitt richtet sich an den speziell für das entsprechende Fachgebiet
interessierten Ingenieur. Er soll es ihm ermöglichen, die für die Problemlösung
gemachten Überlegungen zu verstehen und nachzuvollziehen. Theoretische
Grundlagen sind nur so weit auszuarbeiten, als dies für die Lösung der Aufgabe
nötig ist (keine Lehrbücher schreiben). Die Erkenntnisse aus den theoretischen
Untersuchungen sind wenn immer möglich direkt mit der Problemlösung zu
verknüpfen.

Schlussfolgerungen, Bewertung der Ergebnisse.
Die Schlussfolgerungen bilden zusammen mit der Zusammenfassung die
wichtigsten Abschnitte eines Berichts und sollen daher am sorgfältigsten
ausgearbeitet sein. Die Schlussfolgerungen enthalten eine Zusammenfassung
und Beurteilung der Resultate (Vergleich mit anderen Lösungen, was wurde
erreicht, was nicht, was bleibt noch zu tun, was würde man nun anders tun). In
den Schlussfolgerungen soll auch ein Ausblick auf das weitere Vorgehen bzw.
auf die Bedeutung der erreichten Ergebnisse gegeben werden.

Installationsanleitung vorhanden (inklusive verwendete Entwicklungsumgebung und Werkzeuge), Test-Logs dokumentiert (bei Systemen mit User Interfaces: Dokumentation der Usability Tests)

Ideen für Schlussfolgerung:
* fehlendes Know-How?
* Positives
* negatives
* Vor-/Nachteile Rails? Was wenn ohne Rails gemacht?
\end{comment}

\chapter{Ergebnis}
\label{results}

Im Rahmen der Bachelorarbeit wurde eine Softwarelösung mit zentraler Web-Komponente und verteilten Agenten entwickelt. Die Web-Applikation kann auf den gängigen Browsern gut dargestellt werden\footnote{Getestet auf Firefox 47 und Chrome 51} und kann unter normalen Bedingungen mit den Agenten kommunizieren.

Die Anforderungen aus dem Projektauftrag (siehe Kapitel \ref{sec:anforderung:aufgabenstellung} bezüglich Nummerierung) wurden umgesetzt:

\begin{enumerate}[noitemsep]
    \item Die Applikation kann über den Web-Browser erreicht werden.
    \item Anstehende Updates können auf den verbundenen Systemen eingesehen werden.
    \item Es kann pro System eine Übersicht der ausgeführten Tasks eingesehen werden, worin die durchgeführten Updates enthalten sind.
    \item Pro System läuft ein Agent, welcher mit dem Control Center kommuniziert.
    \item Die Kommunikation mit dem Control Center verläuft verschlüsselt bei richtiger Konfiguration.
    \item Meldungen von \gls{apt} werden weitergereicht und bei Fehlschlagen eines Updates wird dieses besonders gekennzeichnet. Wird ein Neustart benötigt, wird das dem Control Center ebenfalls mitgeteilt.
    \item Die kürzlich durchgeführten Aktivitäten aller Benutzer ist einfach einsehbar und es kann pro Job ein Kommentar vom Auslöser hinzugefügt werden.
    \item [9.] Die Bedienbarkeit wurde durch Usability-Tests bes    tätigt.
    \item [10.] Die Software wurde als Open-Source-Lösung mit der MIT-Lizenz (siehe auch Anhang \ref{sec:license}) entwickelt.
    \item [11.] Sie unterstützt mindestens die Ubuntu-Versionen 12.04, 14.04 und 16.04.
    \item [12.] Als Programmiersprachen wurden hauptsächlich Ruby und Python verwendet.
    \item [13.] Als Datenbank kommt PostgreSQL zum Einsatz.
    \item [14.] Es wurde ein Python-Paket\footnote{Siehe Kapitel \ref{sec:architecture:agent}} erstellt, welches auch mit Puppet verteilt werden kann.
\end{enumerate}

Punkt 8 konnte nicht im vorgesehenen Detailgrad erarbeitet werden. Die durchgeführten Tests zeigten keine Probleme bezüglich der Zuverlässigkeit. Insbesondere das Verhalten bei vielen parallelen Zugriffen wurde nicht spezifisch getestet. Auf der Seite des Control Centers kann mit der maximaler Anzahl Threads in Passenger das Problem entschärft werden.

Der optionale Punkt 15 wurde nicht erreicht. Systeme und Pakete müssen manuell an Gruppen zugewiesen werden. Es wurden jedoch Vorschläge und Mockups dazu erstellt, siehe Kapitel Ausblick (\ref{sec:ausblick:auto_group_assignment}).

\section{Schlussfolgerung}
\label{conclusion}

\gls{upd89} ist grafisch nicht so detailreich ausgearbeitet wie kommerzielle Produkte wie etwa \hyperref[sec:analysis:competition:landscape]{Landscape} und besitzt nicht so viele Features wie grössere Open-Source-Produkte wie \hyperref[sec:analysis:competition:spacewalk]{Spacewalk}. Im Vergleich mit anderen verfügbaren Produkten konzentriert sich die Applikation auf die Kernfunktion, das Aktualisieren von Ubuntu-Systemen. Dies wurde erreicht und funktioniert stabil.

Die Vorgabe, dass Rails benutzt werden sollte, war einerseits ein kleiner Nachteil, da praktisch noch kein Know-How im Team vorhanden war; andererseits war es auch ein Vorteil, da Rails bei der Entwicklung viel Arbeit abnahm. Bei einer Durchführung ohne Rails hätte ein anderes Web-Applikations-Framework verwendet werden müssen, wo das Know-How ebenfalls nicht vorhanden gewesen wäre.

Dass einige Features nicht umgesetzt werden konnten (beispielsweise das \hyperref[sec:ausblick:regelwerk]{Regelwerk} und weitere Features im Kapitel \hyperref[sec:ausblick]{Ausblick}) ist schade, trotzdem konnten alle primären Anforderungen erfüllt werden. Auch dass die Projektpartner interessiert am Projektverlauf waren und eigene Ideen einbrachten, zeugt von einer prinzipiell guten Idee. Ob das Projekt von der Open-Source-Gemeinde angenommen und benutzt wird, bleibt abzuwarten.

\section{Softwaredokumentation} \label{documentation}

In diesem Kapitel wird kurz beschrieben, wie die Software benutzt werden soll. Im Anhang (Kapitel \ref{appendix:documentation}) ist die komplette Anleitung zu finden, so wie sie auf Github hinterlegt ist. Da die öffentlichen Dokumente in Englisch verfasst wurden (Siehe auch \ref{sec:implementation:documentation}), ist dort auch die Softwaredokumentation auf Englisch.

\subsection{\gls{ca}}
Für die Absicherung der Kommunikation zwischen Control Center und Agent verwenden wir Zertifikate, die von einer eigenen \gls{ca} signiert sind (vgl. Kapitel \ref{sec:security}).

\begin{srclst}[label=lst:ca:installation1]{bash}{Setup Easy-RSA}
sudo apt install easy-rsa
make-cadir ca
cd ca
sed -i 's/extendedKeyUsage.*/extendedKeyUsage=serverAuth,clientAuth/g' openssl*.cnf
sed -i 's/export KEY_SIZE=.*/export KEY_SIZE=4096/g' vars
nano vars
\end{srclst}

\begin{srclst}[label=lst:ca:installation2]{bash}{CA Konfiguration}
export KEY_COUNTRY="US"
export KEY_PROVINCE="CA"
export KEY_CITY="SanFrancisco"
export KEY_ORG="Fort-Funston"
export KEY_EMAIL="me@myhost.mydomain"
export KEY_OU="MyOrganizationalUnit"
\end{srclst}

\begin{srclst}[label=lst:ca:installation3]{bash}{Generieren der Zertifikate}
./clean-all
./pkitool --initca
./pkitool --server controlcenter.your-domain.org
./pkitool agent1.your-domain.org
./pkitool agent2.your-domain.org
./pkitool agent3.your-domaon.org
\end{srclst}

\subsection*{Control Center}

\subsubsection*{Vorbedingungen}

Es gelten folgende Voraussetzungen, damit ein reibungsloser Betrieb der Software sichergestellt werden kann:

\begin{itemize}
    \item Git muss installiert sein
    \item Eine \gls{ca} muss aufgesetzt sein
    \item Passenger 5.0.27 / Apache 2.4.7 (Ubuntu 14.04) müssen installiert und konfiguriert sein
    \item Ruby 2.2.3 und Rails 4.2.4 müssen installiert sein
    \item Postgresql-9.3 muss installiert und konfiguriert sein
\end{itemize}

\subsubsection*{Installation}

Sind alle Voraussetzungen erfüllt, kann das Control Center mit folgenden Befehlen installiert werden:

\begin{srclst}[label=lst:cc:installation]{bash}{Installation}
git clone https://github.com/upd89/controlcenter.git
cd controlcenter
bundle install
rake db:create
rake db:migrate
rake db:base_data
\end{srclst}

Zu Testzwecken kann der Server lokal gestartet werden:

\begin{srclst}[label=lst:cc:localtest]{bash}{Startet einen lokalen Server}
rails server
\end{srclst}


\subsection*{Agent}

Der Agent kann auf verschiedene Arten installiert werden. Einerseits wie das Control Center durch das Klonen von GitHub:

\begin{srclst}[label=lst:agent:installation]{bash}{Installation}
apt install python-apt python-daemonize python-configparser
git clone https://github.com/upd89/agent.git
python setup.py build
python setup.py install
update-rc.d upd89 defaults
\end{srclst}

Andererseits aber auch durch pip:

\begin{srclst}[label=lst:agent:installation_by_pip]{bash}{Installation via pip}
pip install upd89
update-rc.d upd89 defaults
\end{srclst}

Anpassungen können in \mintinline{console}{/etc/upd89/config.ini} vorgenommen werden.

Hinweis: Es wird eine direkte Verbindung vom Agent zum Control Center benötigt. Dies bedeutet, dass jeder Agent via HTTPS das Control Center erreichen können muss und umgekehrt.