\begin{comment}
Ergebnisse der Arbeit: was wurde erreicht, was wurde nicht erreicht, Ursachen.
Dieser Abschnitt richtet sich an den speziell für das entsprechende Fachgebiet
interessierten Ingenieur. Er soll es ihm ermöglichen, die für die Problemlösung
gemachten Überlegungen zu verstehen und nachzuvollziehen. Theoretische
Grundlagen sind nur so weit auszuarbeiten, als dies für die Lösung der Aufgabe
nötig ist (keine Lehrbücher schreiben). Die Erkenntnisse aus den theoretischen
Untersuchungen sind wenn immer möglich direkt mit der Problemlösung zu
verknüpfen.

Schlussfolgerungen, Bewertung der Ergebnisse.
Die Schlussfolgerungen bilden zusammen mit der Zusammenfassung die
wichtigsten Abschnitte eines Berichts und sollen daher am sorgfältigsten
ausgearbeitet sein. Die Schlussfolgerungen enthalten eine Zusammenfassung
und Beurteilung der Resultate (Vergleich mit anderen Lösungen, was wurde
erreicht, was nicht, was bleibt noch zu tun, was würde man nun anders tun). In
den Schlussfolgerungen soll auch ein Ausblick auf das weitere Vorgehen bzw.
auf die Bedeutung der erreichten Ergebnisse gegeben werden.

Installationsanleitung vorhanden (inklusive verwendete Entwicklungsumgebung und Werkzeuge), Test-Logs dokumentiert (bei Systemen mit User Interfaces: Dokumentation der Usability Tests)

Ideen für Schlussfolgerung:
* fehlendes Know-How?
* Positives
* negatives
* Vor-/Nachteile Rails? Was wenn ohne Rails gemacht?
\end{comment}

\chapter{Ergebnis}

Im Rahmen der Bachelorarbeit wurde eine Softwarelösung mit zentraler Web-Komponente und verteilten Agenten entwickelt. Die Web-Applikation kann auf den gängigen Browsern gut dargestellt werden\footnote{Getestet auf Firefox 47 und Chrome 51} und kann unter normalen Bedingungen mit den Agenten kommunizieren.

Die Anforderungen aus dem Projektauftrag (siehe \ref{sec:anforderung:aufgabenstellung}) wurden umgesetzt.

\begin{itemize}
    \item Die Applikation kann über den Web-Browser erreicht werden.
    \item Anstehende Updates können auf den verbundenen Systemen eingesehen werden
    \item Es kann pro System eine Übersicht der ausgeführten Tasks eingesehen werden
    \item Pro System läuft ein Agent und die Kommunikation mit dem Control Center verläuft verschlüsselt.
    \item Meldungen von {apt} werden weitergereicht und bei Fehlschlagen eimes Updates wird dieses besonders gekennzeichnet. Benötigte Updates werden dem Control Center ebenfalls mitgeteilt.
    \item Die kürzlich durchgeführten Aktivitäten aller User ist einfach einsehbar und es kann pro Job ein Kommentar vom Auslöser hinzugefügt werden.
\end{itemize}

\xxx[Was ist mit den Empfehlungen bei Neustart etc.?]
\xxx[korrekter und konsistenter Datenstand?]

Die Software wurde als Open-Source-Lösung mit der MIT-Lizenz (siehe auch Kapitel \ref{sec:license}) entwickelt und unterstützt mindestens die Ubuntu-Versionen 12.04, 14.04 und 16.04. Als Programmiersprachen wurden hauptsächlich Ruby und Python verwendet, als Datenbank kommt PostgreSQL zum Einsatz.

\section{Schlussfolgerung}
\xxx

\section{Softwaredokumentation}
\xxx[Anhang/Github-Dingens zusammenfassen und übersetzen (Vorbedingungen nur kurz erwähnen, z.B. CA oder direkte Verbindung benötigt von A zu CC)]

\subsection*{Control Center}



\subsection*{Agent}

