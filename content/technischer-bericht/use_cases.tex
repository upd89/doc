\section{Use Cases}

In diesem Kapitel werden die Anforderungen an die Software in Form von Use-Cases (Anwendungsfällen) dokumentiert. Die Use-Cases wurden aufgrund des Auftrages und weiteren Besprechungen mit dem Industriepartner definiert. Abbildung \ref{fig:usecases} zeigt eine Übersicht über die Use-Cases für den Agent sowie das Control Center.

\begin{figure}
  \centering
    \includegraphics[width=0.9\textwidth]{files/UseCases}
  \caption{Übersicht der Use Cases}
  \label{fig:usecases}
\end{figure}


\subsection*{UC01: Agent deployen}
\label{sec:uc_01}

Der Agent wird auf einem System ausgeliefert. Dabei werden Parameter zur Konfiguration des Agents bei der Installation mitgegeben.


Technische Details:

\begin{itemize}
    \item Das Paket wird als .deb via Puppet ausgeliefert
    \item Konfiguration des Agents enthält: Adresse des Control-Centers und das CA-Zertifikat
\end{itemize}


\subsection*{UC02: Registration von Agent an Control Server}
\label{sec:uc_02}

Ein Agent registriert sich beim Control-Center.

Technische Details:

\begin{itemize}
    \item Registration geschieht über die beim Deployment erhaltene Adresse und dem generierten Zertifikat
    \item Der Agent übermittelt seinen Fully Qualified Domain Name
\end{itemize}

\subsection*{UC03: Ausstehende Updates melden}
\label{sec:uc_03}

Ein Agent aktualisiert die Paketliste. Er prüft, ob Updates anstehen und meldet diese Liste dem Control-Center. Gleichzeitig sendet er auch noch Hostinformationen, damit das Control-Center seine Informationen ggf. aktualisieren kann.
 

Technische Details:

\begin{itemize}
    \item Paketliste wird via apt-Schnittstelle aktualisiert
    \item Hostinformationen enthalten seine OS-Version sowie Hostnamen
\end{itemize}

\subsection*{UC04: Update durchführen}
\label{sec:uc_04}

Ein Agent erhält vom Control-Center einen Task mit einer Liste von vorzunehmenden System-Updates. Er prüft, ob diese bei sich anstehen und führt diese durch. Das Empfangen der Updates wird dem Control-Center bestätigt, wo der Status dieses Tasks auf 'Queued' gesetzt wird.

\subsection*{UC05: Update auslösen}
\label{sec:uc_05}

Das Control-Center hat eine Liste von ausstehenden Updates von einem oder mehreren Systemen erhalten. Diese zeigt es in einer Übersicht an. Ein User mit entsprechender Berechtigungs-Stufe gibt eines oder mehrere dieser Updates frei. Das Control-Center erstellt daraus Tasks, welche der User freigeben kann. Falls die Tasks freigegeben werden, schickt das Control-Center diese an die jeweiligen Systeme.

\subsection*{UC06: Statusreport anzeigen}
\label{sec:uc_06}

Ein Benutzer lässt sich vom Control-Center eine Liste von Systemen und Updates anzeigen. Der User sieht den Zustand jedes Systems und Updates.

\subsection*{UC07: Berechtigungen verwalten}
\label{sec:uc_07}

Ein Administrator kann für Systemgruppen und Paketgruppen Berechtigungs-Stufen setzen, so dass nur User mit dieser Stufe Aktionen auf diesen Gruppen vornehmen können.

\subsection*{UC08: Systemgruppen verwalten}
\label{sec:uc_08}

Ein Administrator kann eine System-Gruppe erstellen, bearbeiten oder löschen. Dies beinhaltet auch das Entfernen oder Hinzufügen von Systemen sowie das Setzen von Berechtigungs-Stufen.

\subsection*{UC09: User verwalten}
\label{sec:uc_09}

Ein Administrator kann neue User erstellen, Berechtigungs-Stufen ändern oder bestehende User löschen.

\subsection*{UC10: Regeln verwalten}
\label{sec:uc_10}

Ein Administrator kann Regeln für Systemgruppen erstellen, ändern und entfernen. Diese Regeln greifen bei der Auswahl der Systeme und Pakete.

\subsection*{UC11: Paketgruppen verwalten}
\label{sec:uc_11}

Ein Administrator kann neue Gruppen für Pakete erstellen, bearbeiten oder löschen. Dies beinhaltet auch das Entfernen oder Hinzufügen von Paketen sowie das Setzen von Berechtigungs-Stufen.
