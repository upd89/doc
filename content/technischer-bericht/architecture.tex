\begin{comment}
Mit begründeten Architekturentscheidungen, Mit Diskussion, wie Q-Attribute sichergestellt wurden (welche Qualität wurde erreicht?), Mit Dokumentation, welche Tests durchgeführt wurden, welche Lösungsoptionen wurde aufgrund von Tests/Experimenten verworfen.
\end{comment}

\section{Architektur}

\xxx[MVC etwas besser/tiefer erklären]

\subsection*{Control Center}

Das \gls{controlcenter} wurde mit Ruby on Rails umgesetzt. Rails ist ein \gls{mvc}-Framework, wodurch grundsätzliche Entscheidungen zur Architektur bereits vorweggenommen wurden.

Das \gls{mvc}-Pattern teilt Applikationen in drei Teile, welche jeweils verschiedene Aufgaben und Verantwortungen haben. Ziel ist, dass die Wiederverwendbarkeit und Trennbarkeit der Komponenten gewährleistet wird.

Ruby on Rails bildet das Pattern bereits in der Dokumentenstruktur ab. Im Verzeichnis /app befinden sich Ordner für Controllers, Models und Views.

\begin{figure}[H]
	\centering
	\includegraphics[width=0.5\linewidth]{files/mvc_structure}
	\caption{MVC-Pattern in Rails-Umgebung}
	\label{fig:tec:mvc}
\end{figure}

Als Rails-Version wurde 4.2 gewählt, obwohl Rails 5 im Mai 2016 als Release Candidate erschienen wäre\footnote{\purl{http://weblog.rubyonrails.org/2016/5/6/this-week-in-rails-railsconf-recap-rails-5-0-rc-1-is-out/}} und Features wie z.B. WebSockets beinhaltet hätte. Es wurde aber gegen den Gebrauch der neuen Version entschieden, da sie zum Zeitpunkt der Evaluation noch nicht aus dem Beta-Stadium heraus war und die Kompatibilität mit verschiedenen Ruby-Gems nicht immer gewährleistet ist.

\xxx[source richtig zitieren!]
\xxx[eventuell Decision-Block?]

\subsubsection*{Gems}

Rails ist ein Gem für Ruby. Gems sind Pakete für den Paketmanager von Ruby (RubyGems), womit für eine Ruby-Applikation vorgefertigte Bibliotheken und weitere Applikationen eingebunden werden können.

Unter anderem wurden bei \gls{upd89} folgende Gems verwendet:

\begin{labeling}{font\textunderscore awesome\textunderscore rails}
    \item [sass-rails] Kompiliert Sass- und Scss-Files nach CSS
    \item [coffee-rails] Ermöglicht das Benutzen von Coffeescript\footnote{\purl{http://coffeescript.org/}}
    \item [jquery-rails] jQuery wird für DOM-Manipulationen und \gls{ajax} benutzt
    \item [rspec-rails] Testing-Framework für Rails
    \item [factory\textunderscore girl\textunderscore rails] Stellt Fixtures für Unit-Testing zur Verfügung
    \item [rubocop] Prüft den Stil des Codes gemäss den Ruby Code-Guidelines
    \item [faker] Ermöglicht einfaches 'Faken' von Daten für Unit-Testing
    \item [better\textunderscore errors] Hilfreich bei der Entwicklung anstelle der standardmässig eher unspektakulären Fehlermeldungen
    \item [rake] 'Software-Management Task Tool', hilft beim Ausführen von Tasks wie z.B. dem Erstellen von Default-Daten für die Datenbank
    \item [font\textunderscore awesome\textunderscore rails] Bindet die Icons von FontAwesome ein
    \item [will\textunderscore paginate] Ermöglicht \gls{pagination} von ActiveRecord-Einträgen
    \item [sucker\textunderscore punch] Asynchrone Hintergrund-Aufträge, wird für das Senden von Tasks an die \glspl{agent} benötigt
    \item [faraday] Bibliothek für HTTP-Zugriffe mit HTTPS-Support
    \item [sorcery] Benutzer-Authentifizierung und Login-Funktionalität
    \item [filterrific] Unterstützung für Filter und Sortiermöglichkeiten
    \item [cancancan] Berechtigungen auf Aktions-Ebene (\gls{crud})
\end{labeling}

\subsubsection*{Active Record}

\xxx[https://en.wikipedia.org/wiki/Active\textunderscore record\textunderscore pattern]

\subsection*{Agent}

\xxx

\subsection*{API}

\xxx[Nicht details, aber evtl. sequenzdiagramm und so]