\begin{comment}
2.1.2 Abstract
Ein Abstract ist eine rein textuelle kurze Zusammenfassung der Arbeit. Der Abstract ist für die Recherche in grossen Dokumentensammlungen geeignet. Er umfasst nie mehr als eine Seite, typisch sogar nur etwa 200 Worte (etwa 20 Zeilen).
Der Begriff ‚Kurzfassung’ ist zuwenig genau definiert; er soll wenn möglich vermieden werden.


“Der Abstract richtet sich an den Spezialisten auf dem entsprechenden Gebiet und beschreibt daher in erster Linie die (neuen, eigenen) Ergebnisse und Resultate der Arbeit. Es umfasst nie mehr als eine Seite, typisch sogar nur etwa 200 Worte (etwa 20 Zeilen). Es sind keine Bilder zu verwenden.” (Anleitung: Dokumentation Studien- und Bachelorarbeiten)


\end{comment}


\phantomsection
\addcontentsline{toc}{chapter}{Abstract}
\chapter*{Abstract}


Das Unternehmen Nine Internet Solutions AG betreibt Linux-Server für Unternehmen. Die Server müssen bezüglich Sicherheit und Aktualisierungen fortlaufend auf dem neuesten Stand gehalten werden. Dies wird durch einen Mitarbeiter erledigt, welcher im Durchschnitt einen Tag pro Woche damit verbringt. Aufgrund der grossen Anzahl Server und Pakete sowie Abhängigkeiten zwischen verschiedenen Arten von Servern ist es schwierig, den Überblick über die installierten Versionen und anstehenden Updates zu behalten.

Als Ergebnis dieser Arbeit wurde eine Web-Applikation entwickelt, mit welcher Systemadministratoren Überblick über den Zustand der zu verwaltenden Systeme bezüglich der Aktualisierungen gewinnen und Updates auf Systemen in Auftrage geben können. Diese Updates werden anschliessend automatisch auf den Servern installiert.

Das zentrale Kontrollcenter wurde mit Ruby on Rails entwickelt, die Agenten auf den Servern mit Python. Die Agenten kommunizieren mit dem Kontrollcenter asynchron und via TLS über eine API. Auf dem Kontrollcenter kommt PostgreSQL als Datenbank zum Einsatz, wo alle Informationen über die Systeme verwaltet werden.

\xxx[Mehr zur Umsetzung?]

\xxx[Nutzen für den obigen Mitarbeiter? Zeitersparnis, Überblick?]


\bigskip
Webseite: \purl{https://upd89.org}


\glsresetall
