\begin{comment}
2.1.3 Management Summary und Web-Publikation
Das Management Summary soll 2-5 Seiten umfassen sowie eine bis zwei Figuren enthalten. Es richtet sich an den „gebildete Laien“ auf dem Gebiet und beschreibt daher in erster Linie die (neuen und eigenen) Ergebnisse und Resultate der Arbeit. Die Sprache soll knapp, klar und stark untergliedert sein.
Grundlage für das Management Summary kann der Broschüren-Eintrag sein, den die Abteilung bei Diplomarbeiten jeweils früh verlangt, um eine Broschüre zu drucken. Das Management Summary dient als Vorlage für eine allfällige Web-Publikation.
Das Abstract und das Management Summary werden - zeitlich gesehen - gegen Schluss der Arbeit geschrieben und bilden zusammen mit den Schlussfolgerungen im technischen Bericht den am häufigsten gelesenen Teil der Arbeit. Diese Dokumente sollen daher am Sorgfältigsten ausgearbeitet sein.
Die folgenden Stichworte sollen die typische Struktur illustrieren, wobei die genaue Ausführung jeweils auf die spezifischen Bedürfnisse und Randbedingungen eines Projekts anzupassen ist. Diese Struktur kann auch für die Präsentation der Arbeit als "Richtschnur" dienen.
1. Ausgangslage
Warum machen wir das Projekt?
Welche Ziele wurden gesteckt (Kann-Ziele, Muss-Ziele)
Was machen andere / welche ähnlichen Arbeiten gibt es zum Thema?
Vorgehen: Was wurde gemacht? In welchen Teilschritten?
Risiken der Arbeit?
Wer war involviert (Durchführung, Entscheide usw.)?
Was konnte von anderen verwendet werden?
2. Ergebnisse
Was ist das Resultat?
Bewertung der Resultate, was ist Neuartig an der Arbeit?
Zielerreichung bezüglich Kann-/Muss-Zielen
Abweichungen (positiv und negativ) und kurze Begründung dafür
(Externe) Kosten der Arbeit?
Was ist der Nutzen (quantifizierbar/nicht quantifizierbar)?
3. Ausblick
Was hat man mit Durchführung des Projekts gelernt?
Verbleibende Probleme, (zukünftige) Gegenmassnahmen bez. Risiken
Was würde man anders machen, was ist weiter zu tun

Überschriften (Unterkapitel) ohne Nummerierung einsetzen!
\end{comment}


\phantomsection
\addcontentsline{toc}{chapter}{Management Summary}

\chapter*{Management Summary}
\glsresetall

\section*{Ausgangslage}

\xxx

\section*{Ziele}

\xxx

\section*{Ergebnisse}

\xxx

\section*{Ausblick}

\xxx

