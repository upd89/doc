% label prefix for this part: pm
\part{Projektmanagement}

\chapter{Vorgehensmodell}

\xxx

\chapter{Rollen und Verantwortlichkeiten}

\xxx

\chapter{Risiken}\label{sec:risiken}

Um die potentiellen Risiken während des Projekts zu sammeln, wurde eine  Risiko-Checkliste verwendet. Für jedes Risiko wurde ein zusätzlicher Aufwand in Stunden geschätzt, welcher anfallen würde wenn das Risiko eintritt. Multipliziert mit der geschätzten Eintrittswahrscheinlichkeit ergab dies einen gewichteten Mehraufwand. Dieser wurde in 5 Kategorien (sehr gering bis sehr schwer) eingestuft.

Ziel war es, alle Risiken bis Ende Elaborations-Phase entweder komplett auszuschliessen oder soweit zu Entschärfen, dass beim Eintreten Sicherheitsmassnahmen greifen und keine grosse Verzögerung oder anderen Schaden verursachen.

Es wurden nur für das Projekt relevante Risiken analysiert. Risiken wie "Krankheit" wurden absichtlich nicht notiert, da es hier keine sinnvollen Präventionsmassnahmen gibt.

\begin{landscape}
	\begin{table}[H]
		\centering
		\includegraphics[[width=0.9\textwidth,keepaspectratio]{Risikoanalyse.pdf}
		\caption{Alle berücksichtigten Risiken}
		\label{tab:risikoanalyse}
	\end{table}
\end{landscape}

\section{Kritische Risiken}

{Die besonders kritischen Risiken werden hier kurz erläutert und genauer beschrieben.}

\projectrisk{R01}{Regelwerk zwischen Updates und Systemen zu komplex}
{Es war von Beginn an nicht klar, wie genau das Regelwerk genau aussehen soll, welches die Kombinationen von Systemen und Updates einschränken soll. Im schlimmsten Fall können es beliebig tiefe Abhängigkeiten sein, welche über mehrere Gruppen hinweg geprüft werden müssen.}
{In der Elaborations-Phase sollte so gut wie möglich geklärt werden, in welcher Form das Regelwerk entstehen soll. Sobald sich herausstellt, dass die gewählte Tiefe der Umsetzung zu viel Aufwand verursacht, wird mit dem Auftraggeber entschieden, ob die Anforderung gegebenfalls eingegrenzt werden kann.}


\projectrisk{R02}{Komplexes Permission-System für User}
{Wie beim Regelwerk zwischen Updates und Systemen können die Berechtigungen der Benutzer beliebig fein konzeptioniert werden. Dadurch müssen nicht nur die Berechtigungen selbst geprüft werden, sondern auch Abhängigkeiten unter den Berechtigungen - es ist sinnlos, wenn man einem Benutzer das Updaten einer spezifischen Gruppe verbietet, aber das Erstellen und Bearbeiten einer Gruppe erlaubt ist.}
{Wiederum soll in der Elaboration-Phase mit dem Auftraggeber das Rechtesystem vereinbart werden, so dass die Machbarkeit gewährleistet ist. Falls dies nicht gelingt oder sich als schwerer als geplant entpuppt, sollen wenn möglich die Anforderungen eingeschränkt werden. Falls es keine brauchbare Lösung gibt, muss auf das Rechtesystem verzichtet werden.}

\chapter{Infrastruktur}

\xxx

\xxx[similar chart would be sweet]

\begin{figure}[H]
	\centering
	\includegraphics[width=\linewidth]{fig/entwicklungsumgebung}
	\caption{Entwicklungsumgebung}
	\label{fig:pm:entwicklungsumgebung}
\end{figure}

\section{Projektmanagement}

\chapter{Qualitätsmanagement}

\xxx

\subimport{sprints/}{sprints.tex}
