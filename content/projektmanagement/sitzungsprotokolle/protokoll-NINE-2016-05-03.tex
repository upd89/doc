\documentclass[class=scrbook,crop=false]{standalone}

\usepackage{../../../style}
%------------------------------------------------------------------------------
% Glossary
%------------------------------------------------------------------------------

\newglossaryentry{test}{name=TestTest, description={ist ein Test-Eintrag},url={http://www.test.com/}}

%------------------------------------------------------------------------------
% Acronyms
%------------------------------------------------------------------------------

\newacronym{avt}{AVT}{Arbeiten-Verwaltungs-Tool der \acrshort{hsr}}

\newacronym{ba}{BA}{Bachelorarbeit}

\newacronym[url={http://hsr.ch/}]{hsr}{HSR}{Hochschule für Technik Rapperswil}

\newacronym{ci}{CI}{Continuous Integration}

\newacronym{upd89}{UPD89}{Update Nine.ch}

\newacronym{vm}{VM}{Virtuelle Maschine}

% names & full names
\newcommand{\ubo}{U.\ Bosshard\xspace}
\newcommand{\ubos}{Ueli Bosshard\xspace}

\newcommand{\pch}{P.\ Christen\xspace}
\newcommand{\pchr}{Philipp Christen\xspace}

\newcommand{\proff}{Prof.\ Farhad Mehta\xspace}
\newcommand{\prof}{Prof.\ F.\ Mehta\xspace}

% \mytable{cols}{content}{caption}{lbl}
\newcommand{\mytable}[4]{
	\begin{table}[H]
		\centering
		\begin{tabularx}{\textwidth}{#1}
			\toprule
			#2
			\bottomrule
		\end{tabularx}
		\caption{#3}
\label{tab:#4}
	\end{table}
}


% uses todonotes
\newcommand{\xxx}[1][]{
	\ifthenelse{\equal{#1}{}}{\todo[inline]{TODO}}{\todo[inline]{TODO:\ #1}}
}

% pretty url without http:// or https://
\newcommand{\purl}[1]{%
	\href{#1}{\StrBehind{#1}{://}}%
}
\newcommand{\shellcmd}[1]{\\\indent\indent\texttt{\footnotesize\# #1}\\}

\newcommand*{\zeroOfThree}{\FiveStar\FiveStar\FiveStarOpen}
\newcommand*{\oneOfThree}{\FiveStar\FiveStarOpen\FiveStarOpen}
\newcommand*{\twoOfThree}{\FiveStar\FiveStar\FiveStarOpen}
\newcommand*{\threeOfThree}{\FiveStar\FiveStar\FiveStar}

\newcommand*\cleartoleftpage{%
  \clearpage
  \ifodd\value{page}\hbox{}\newpage\fi
}


% Remove section numbering
\renewcommand*\thesection{}

% Remove spacing between section numbering and section titl
\makeatletter
\renewcommand*{\@seccntformat}[1]{\csname the#1\endcsname}
\makeatother

\begin{document}
	
    \section{Meeting Nine.ch, 3. Mai 2016}
    
    \begin{tabular}{ll}
        \textbf{Datum} & 3.5.2016 \\
        \textbf{Zeit} & 16:00 \textendash 17:00 Uhr \\
        \textbf{Ort} & \acs{hsr} \\
        \textbf{Anwesende} & \sasie \\ & \ubos \\ & \pchr
    \end{tabular}
    
    \subsection*{Traktanden}
    
    \begin{itemize}
        \item Abschluss Sprint 3
        \item Planung Sprint 4
        \item Aktuelle Fragen
        \begin{itemize}
            \item Löschen bei bestehenden Assoziationen verhindern
        \end{itemize}
    \end{itemize}

		\subsection*{Fragen}
    
    \subsection*{Protokoll}
    
	\begin{itemize}
        \item Demo: Tasks erstellen + an Agent senden
        \item Sprint 3 im Verzug, aber Hauptziele erreicht
        \item kein SQLite, sondern eigener Key-Value-Store: Wenig Abhängigkeiten sind nice
        \item Resultat von Apt muss noch ausgelesen werden. Nine.ch benutzt apt-hook und parsed den Output
        \item Sprint 4: Certificate, Security, etc.
        \begin{itemize}
            \item Device sei ein "Monster", sehr umfangreich. Besser: Sorcery
            \item CanCanCan (NICHT CanCan) ist super (Tipp: load\textunderscore and\textunderscore authorize\textunderscore resource im Header verwenden für Automatische Implementation von Standard-Aktionen)
            \item Responders-Gem anstatt den üblichen JSON/HTML-Renderings
            \item eventuell auch Rolify
            \item Tool für eigene CA: easy-rsa
        \end{itemize}
        \item Für Lösch-Hinweis bei bestehender Verbindung: Abfrage von record.destroy und entsprechende Nachricht anzeigen
        \item Lasttests: jMeter
        \item Unbedingt Controller-Actions testen. Bei Bedarf kann Phil von Nine.ch in Slack-Channel hinzugefügt werden
        \item Server: CA-Check direkt von Apache handeln lassen --> zweiter vhost für API
        \item Tipp: Controller vereinfachen. "load and authorize", "cancancan" -> kontroller verkürzen, responders -> vereinfachen
        \item Nächstes Meeting 17.5.
    \end{itemize}
    

\end{document}