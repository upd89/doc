\documentclass[class=scrbook,crop=false]{standalone}

\usepackage{../../../style}
%------------------------------------------------------------------------------
% Glossary
%------------------------------------------------------------------------------

\newglossaryentry{test}{name=TestTest, description={ist ein Test-Eintrag},url={http://www.test.com/}}

%------------------------------------------------------------------------------
% Acronyms
%------------------------------------------------------------------------------

\newacronym{avt}{AVT}{Arbeiten-Verwaltungs-Tool der \acrshort{hsr}}

\newacronym{ba}{BA}{Bachelorarbeit}

\newacronym[url={http://hsr.ch/}]{hsr}{HSR}{Hochschule für Technik Rapperswil}

\newacronym{ci}{CI}{Continuous Integration}

\newacronym{upd89}{UPD89}{Update Nine.ch}

\newacronym{vm}{VM}{Virtuelle Maschine}

% names & full names
\newcommand{\ubo}{U.\ Bosshard\xspace}
\newcommand{\ubos}{Ueli Bosshard\xspace}

\newcommand{\pch}{P.\ Christen\xspace}
\newcommand{\pchr}{Philipp Christen\xspace}

\newcommand{\proff}{Prof.\ Farhad Mehta\xspace}
\newcommand{\prof}{Prof.\ F.\ Mehta\xspace}

% \mytable{cols}{content}{caption}{lbl}
\newcommand{\mytable}[4]{
	\begin{table}[H]
		\centering
		\begin{tabularx}{\textwidth}{#1}
			\toprule
			#2
			\bottomrule
		\end{tabularx}
		\caption{#3}
\label{tab:#4}
	\end{table}
}


% uses todonotes
\newcommand{\xxx}[1][]{
	\ifthenelse{\equal{#1}{}}{\todo[inline]{TODO}}{\todo[inline]{TODO:\ #1}}
}

% pretty url without http:// or https://
\newcommand{\purl}[1]{%
	\href{#1}{\StrBehind{#1}{://}}%
}
\newcommand{\shellcmd}[1]{\\\indent\indent\texttt{\footnotesize\# #1}\\}

\newcommand*{\zeroOfThree}{\FiveStar\FiveStar\FiveStarOpen}
\newcommand*{\oneOfThree}{\FiveStar\FiveStarOpen\FiveStarOpen}
\newcommand*{\twoOfThree}{\FiveStar\FiveStar\FiveStarOpen}
\newcommand*{\threeOfThree}{\FiveStar\FiveStar\FiveStar}

\newcommand*\cleartoleftpage{%
  \clearpage
  \ifodd\value{page}\hbox{}\newpage\fi
}


% Remove section numbering
\renewcommand*\thesection{}

% Remove spacing between section numbering and section titl
\makeatletter
\renewcommand*{\@seccntformat}[1]{\csname the#1\endcsname}
\makeatother

\begin{document}
	
    \section*{Meeting Nine.ch, 8. März 2016}
    
    \begin{tabular}{ll}
        \textbf{Datum} & 8.3.2016 \\
        \textbf{Zeit} & 16:00 \textendash 17:00 Uhr \\
        \textbf{Ort} & \acs{hsr} \\
        \textbf{Anwesende} & \sasie \\ & \rulrich \\ & \ubos \\ & \pchr
    \end{tabular}
    
    \subsection*{Traktanden}
    
    \begin{itemize}
        \item Aufgabenstellung
        \begin{itemize}
            \item OK des Kunden
        \end{itemize}
        \item Vorschlag Projektplan
        \begin{itemize}
            \item Meilensteine festlegen
        \end{itemize}
        \item Use-Cases
        \item Abläufe/Mockups
    \end{itemize}
    
    \subsection*{Fragen}
    
    \begin{itemize}
        \item Ruby interface to apt-pkg, Goal of this project is to have a proper Ruby binding to APT like in Python. - Currently install, remove packages commands are not implemented. (Keine Frage)
        \item Debian 6 durch Debian 7 ersetzen? -> Debian nicht mehr erwähnen
        \item Client-Side JS? Präferenzen? Angular ok? -> ok
        \item TDD? In welcher Form? -> keine Vorgabe durch nine
        \item Reboot-required erkennbar, aber nur durch Download der debs + grep nach "requires-reboot". Frage: Handling der Pakete? Caching lokal OK? -> output von apt genügt
        \begin{itemize}
            \item Grösserer Aufwand wenn im Vorherein erkennen, ob reboot required. Im Nachhinein simpel durch /var/run/...
        \end{itemize}
    \end{itemize}
    
    \subsection*{Protokoll}
    
    \subsubsection*{Entscheide}
    
    \begin{itemize}
        \item Debian kann aus der Aufgabenstellung gestrichen werden
        \item Sitzungen alle 2 Wochen bei nine, jeweils Dienstags um 16:00 Uhr
        \item Aufgabenstellung/Projektauftrag OK
    \end{itemize}
    
    \subsubsection*{Tipps}
    
	\begin{itemize}
        \item Bei der Festlegung der verwendeten Versionen auf Supportdauer achten (Bsp. Rails LTS)
        \item Python-Modul 'apt' verwendet C-Bindings, allenfalls können die verwendet werden
        \item Angular erhöht Komplexität unter Umständen unnötig, da vieles mit Rails bereits möglich ist
        \item TDD: Tests sind sinnvoll, nine achtet aber primär auf Nachhaltigkeit
    \end{itemize}
    
    \subsection*{Architektur}
    
    \subsubsection*{Abläufe aus der Praxis}
    
    \begin{itemize}
        \item unkritische Updates auf unkritischen Systemen verteilen
        \item weitere Gruppen von Updates auf Gruppe von Systemen verteilen
        \item 1 System updaten (mit Auswahl aller Updates)
        \item 1 Patch möglichst schnell ausbreiten (selten)
    \end{itemize}
    
    \subsubsection{Gruppierung der Systeme}
    
    \begin{itemize}
        \item Sinnvolle Gruppen: Default, HA-Systeme, DB-Systeme (z.B. falls DB installiert)
        \item Standard-Gruppe für neue Systeme?
        \item Info/Konfiguration von Agent um Gruppenzuordnung zu steuern?
        \begin{itemize}
            \item Gefahr, dass CC von Agent abhängig ist.
        \end{itemize}
        \item Systeme müssen nachträglich anderer Gruppe zugeordnet werden können
        \item Erstellen/Löschen/Ändern von anderen Gruppen trotzdem sinnvoll
    \end{itemize}
    
    \subsubsection{Gruppierung der Pakete}
    
    \begin{itemize}
        \item Unbekannte Pakete werden in Dashbord angezeigt und können dort in eine (bzw mehrere?) Paket-Gruppe eingeteilt werden
    \end{itemize}
    
    \subsubsection{Update-Freigabe}
    
    IST:
    
    \begin{itemize}
        \item Erst unkritische Systeme und unkritische Pakete, dann zweiter/dritter Durchlauf mit immer "kritischeren" Kombinationen bis alles aktualisiert wurde
        \item Erst grosse Masse von unkritischen Kombinationen, danach Spezialfälle
        \item HA-Systeme werden separat aktualisiert, da sehr kritisch
        \item DB-Systeme in einem eigenen Step um verwundbare Fläche (Unterbruchsdauer) klein zu halten (-> Replikation)
        \item Logik basiert derzeit auf Blacklists
        \item Blacklist ist Pattern-Matching von Paket-Namen (Kernel, MySQL, Postgres, ...)
    \end{itemize}
    
    SOLL:
    
    \begin{itemize}
        \item Systemgruppe auswählen
        \item Paketgruppe auswählen
        \item Update-Auswahl anpassen/bestätigen und zur Durchführung freigeben
        \item Auftrag an Agents verteilen
        \item Feedback der Agents anschliessend in Control-Center (in Progress, Erfolgreich, Fehlgeschlagen)
    \end{itemize}
    
    \subsubsection{Berechtigungen}
    
    Rollen:
    
    \begin{itemize}
        \item Admin: Updates freigeben
        \item jeder: Logs/Reports einsehen (Read-Only)
        \item ev weitere Rolle: Unkritische Updates freigeben
    \end{itemize}
    
    \subsubsection{Sonstiges}
    
    \begin{itemize}
        \item Information über empfohlenen System-Neustart aus apt in Control Center anzeigen. Info über benötigte Neustarts von Services nicht notwendig.
        \item Gruppierung nach OS-Version (Ubuntu 12.04, 14.04, 16.04)
        \item Abgrenzung: System-Reboot nicht durch unseren Agent
        \item Optional/Später: Zeitliche Steuerung der Updates (Wartungsfenster)
        \item Optional/Später: Informationen über installierte Pakete in Control Center verfügbar (Sinnvoll?)
        \item Optional/Später: Dry-Run
        \begin{itemize}
            \item Prozentsatz oder bestimmte Anzahl von zufällig ausgewählten Systemen für Dry-Run angeben
            \item Durchsicht der Logfiles (Output von Apt)
            \item Anschliessend "scharfe" Durchführung mit selben Parametern (bedingt eventuell Speichern von Runs)
        \end{itemize}
        \item Name "UPD89" für die BA OK.
    \end{itemize}
    
    \subsection*{Tasks bis zur nächsten Sitzung}
    
    \begin{enumerate}
        \item nine frühzeitig über Infrastruktur-Bedarf informieren (z.B. für Testing)
    \end{enumerate}


\end{document}