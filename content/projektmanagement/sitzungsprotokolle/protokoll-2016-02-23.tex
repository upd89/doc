\documentclass[class=scrbook,crop=false]{standalone}

\usepackage{../../../style}
%------------------------------------------------------------------------------
% Glossary
%------------------------------------------------------------------------------

\newglossaryentry{test}{name=TestTest, description={ist ein Test-Eintrag},url={http://www.test.com/}}

%------------------------------------------------------------------------------
% Acronyms
%------------------------------------------------------------------------------

\newacronym{avt}{AVT}{Arbeiten-Verwaltungs-Tool der \acrshort{hsr}}

\newacronym{ba}{BA}{Bachelorarbeit}

\newacronym[url={http://hsr.ch/}]{hsr}{HSR}{Hochschule für Technik Rapperswil}

\newacronym{ci}{CI}{Continuous Integration}

\newacronym{upd89}{UPD89}{Update Nine.ch}

\newacronym{vm}{VM}{Virtuelle Maschine}

% names & full names
\newcommand{\ubo}{U.\ Bosshard\xspace}
\newcommand{\ubos}{Ueli Bosshard\xspace}

\newcommand{\pch}{P.\ Christen\xspace}
\newcommand{\pchr}{Philipp Christen\xspace}

\newcommand{\proff}{Prof.\ Farhad Mehta\xspace}
\newcommand{\prof}{Prof.\ F.\ Mehta\xspace}

% \mytable{cols}{content}{caption}{lbl}
\newcommand{\mytable}[4]{
	\begin{table}[H]
		\centering
		\begin{tabularx}{\textwidth}{#1}
			\toprule
			#2
			\bottomrule
		\end{tabularx}
		\caption{#3}
\label{tab:#4}
	\end{table}
}


% uses todonotes
\newcommand{\xxx}[1][]{
	\ifthenelse{\equal{#1}{}}{\todo[inline]{TODO}}{\todo[inline]{TODO:\ #1}}
}

% pretty url without http:// or https://
\newcommand{\purl}[1]{%
	\href{#1}{\StrBehind{#1}{://}}%
}
\newcommand{\shellcmd}[1]{\\\indent\indent\texttt{\footnotesize\# #1}\\}

\newcommand*{\zeroOfThree}{\FiveStar\FiveStar\FiveStarOpen}
\newcommand*{\oneOfThree}{\FiveStar\FiveStarOpen\FiveStarOpen}
\newcommand*{\twoOfThree}{\FiveStar\FiveStar\FiveStarOpen}
\newcommand*{\threeOfThree}{\FiveStar\FiveStar\FiveStar}

\newcommand*\cleartoleftpage{%
  \clearpage
  \ifodd\value{page}\hbox{}\newpage\fi
}


% Remove section numbering
\renewcommand*\thesection{}

% Remove spacing between section numbering and section titl
\makeatletter
\renewcommand*{\@seccntformat}[1]{\csname the#1\endcsname}
\makeatother

\begin{document}
	
	\section*{Projektsitzung 23. Februar 2016}
	
	\begin{tabular}{ll}
		\textbf{Datum} & 23.02.2016 \\
		\textbf{Zeit} & 15:10 \textendash 16:10 Uhr \\
		\textbf{Ort} & \acs{hsr} \\
		\textbf{Anwesende} & \proff \\ & \ubos \\ & \pchr
	\end{tabular}
	
	\subsection*{Traktanden}
	\begin{itemize}
		\item Aufgabenstellung
		\item Infrastruktur
		\item Termin für Sitzung
		\item Sonstiges
	\end{itemize}
	
	\subsection*{Protokoll}

	\subsubsection*{Aufgabenstellung}
	
    Die abgegebene Aufgabenstellung wurde zusammen untersucht. Sie enthält noch einige Unklarheiten und Repetition, welche bis zum nächsten Meeting überarbeitet werden.

	\subsubsection*{Infrastruktur}
	
	Bestellung für zwei virtuelle Rechner:
	
	\begin{itemize}
		\item Linux-Image, 4 GB RAM, 50 GB Disk
		\item Linux-Image, Standard-Ausstattung
	\end{itemize}
	
    Im späteren Projektverlauf kann eine grössere Testing-Phase durchgeführt werden. Es ist möglich, dass das INS dafür Ressourcen/Infrastruktur zur Verfügung stellen kann. Dazu muss vom Team ein Testplan erstellt werden --> Mail an Herrn Stettler, CC an Herrn Mehta.

    \subsubsection*{Termin für Sitzung}

    Die Sitzungen mit dem Betreuer finden zu Beginn ein Mal pro Woche statt, später alle zwei Wochen. Der Termin ist ab der dritten Woche fix, vermutlich am Montag. Hr. Mehta wird den Termin für die zweite Woche noch bekannt geben.

    Die Agenda soll 24h vor der Sitzung als PlainText per Mail zu verschicken.

    Vorschlag: Agenda und Protokoll in Wiki führen, damit sie von jedem angepasst werden können.


	\subsubsection*{Sonstiges}
	
	Sprache für die Dokumentation: Deutsch oder Englisch ist egal, frei wählbar
	
Der Report soll eine Abgrenzung zu / einen Vergleich mit anderen Lösungen beinhalten: Puppet, Chef, Vagrant, Windows Update Service.
Hr. Mehta schickt den Bewertungsbogen

    \subsection*{Tasks bis zur nächsten Sitzung}
    
    \begin{enumerate}
        \item Aufgabenstellung überarbeiten
        \begin{enumerate}
            \item Word statt GDocs wegen Formatierung
            \item Technische Anforderungen “oben” erwähnen
            \item Kernproblem ist nicht die Performance, sondern der Aufwand der anfällt
            \item Prios 1+2 MUSS, Prio 3 Nice-to-have
            \item “Regeln” genauer spezifizieren
        \end{enumerate}
        \item Lizenz abklären für Open-Source
        \item Entwurf Projektplan
        \item Nine.ch kontaktieren, wieviel “Kontakt” sie wollen
        \item Sprache für Dokumente definieren
    \end{enumerate}


\end{document}



