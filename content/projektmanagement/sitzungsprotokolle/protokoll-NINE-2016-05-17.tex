\documentclass[class=scrbook,crop=false]{standalone}

\usepackage{../../../style}
%------------------------------------------------------------------------------
% Acronyms
%------------------------------------------------------------------------------

\newacronym{avt}{AVT}{Arbeiten-Verwaltungs-Tool der \acrshort{hsr}}

\newacronym{ba}{BA}{Bachelorarbeit}

\newacronym[url={http://hsr.ch/}]{hsr}{HSR}{Hochschule für Technik Rapperswil}

\newacronym{ci}{CI}{Continuous Integration}

\newacronym{upd89}{UPD89}{Update Nine.ch}

\newacronym{vm}{VM}{Virtuelle Maschine}

\newacronym{wsus}{WSUS}{Windows Server Update Services}
% names & full names
\newcommand{\ubo}{U.\ Bosshard\xspace}
\newcommand{\ubos}{Ueli Bosshard\xspace}

\newcommand{\pch}{P.\ Christen\xspace}
\newcommand{\pchr}{Philipp Christen\xspace}

\newcommand{\proff}{Prof.\ Farhad Mehta\xspace}
\newcommand{\prof}{Prof.\ F.\ Mehta\xspace}

% \mytable{cols}{content}{caption}{lbl}
\newcommand{\mytable}[4]{
	\begin{table}[H]
		\centering
		\begin{tabularx}{\textwidth}{#1}
			\toprule
			#2
			\bottomrule
		\end{tabularx}
		\caption{#3}
\label{tab:#4}
	\end{table}
}


% uses todonotes
\newcommand{\xxx}[1][]{
	\ifthenelse{\equal{#1}{}}{\todo[inline]{TODO}}{\todo[inline]{TODO:\ #1}}
}

% pretty url without http:// or https://
\newcommand{\purl}[1]{%
	\href{#1}{\StrBehind{#1}{://}}%
}
\newcommand{\shellcmd}[1]{\\\indent\indent\texttt{\footnotesize\# #1}\\}

\newcommand*{\zeroOfThree}{\FiveStar\FiveStar\FiveStarOpen}
\newcommand*{\oneOfThree}{\FiveStar\FiveStarOpen\FiveStarOpen}
\newcommand*{\twoOfThree}{\FiveStar\FiveStar\FiveStarOpen}
\newcommand*{\threeOfThree}{\FiveStar\FiveStar\FiveStar}

\newcommand*\cleartoleftpage{%
  \clearpage
  \ifodd\value{page}\hbox{}\newpage\fi
}


% Remove section numbering
\renewcommand*\thesection{}

% Remove spacing between section numbering and section titl
\makeatletter
\renewcommand*{\@seccntformat}[1]{\csname the#1\endcsname}
\makeatother

\begin{document}
	
    \section{Meeting Nine.ch, 17. Mai 2016}
    
    \begin{tabular}{ll}
        \textbf{Datum} & 17.5.2016 \\
        \textbf{Zeit} & 16:00 \textendash 17:00 Uhr \\
        \textbf{Ort} & \acs{hsr} \\
        \textbf{Anwesende} & \sasie \\ & \rulrich \\ & \ubos \\ & \pchr
    \end{tabular}
    
    \subsection*{Traktanden}
    
    \begin{itemize}
        \item Abschluss Sprint 4
        \begin{enumerate}
            \item Kommunikation via TLS mit eigener CA
            \item Automatisches Updaten
            \item Permission via Cancancan
            \item Login via Sorcery
        \end{enumerate}
        \item Planung Sprint 5
        \begin{enumerate}
            \item Reporting
            \item Dashboard
            \item Verbesserungen allgemein
        \end{enumerate}
    \end{itemize}

	\subsection*{Fragen}
	
	\begin{itemize}
        \item Dashboard: Technologien, Empfehlungen?
        \begin{enumerate}
            \item dashing
            \item admin LTE
        \end{enumerate}
        \item Testing mit Sorcery?
        \item Deliverables - fork? Paket?
    \end{itemize}
    
    \subsection*{Protokoll}
    
	\begin{itemize}
        \item Präsentation Frontend-Änderungen (Login, Permissions, System-/Package-Views, etc.)
        \item Präsentation Update-Prozess mit Demo
        \item Sprintziele erreicht
        \item Sprint 5 i.O., keine Bemängelungen
        \item Dashboard: kein Gem o.Ä. verwenden, mit eigener Lösung arbeiten da Daten bereits vorhanden sind
        \item Testing mit Sorcery: Samuel wird nachschauen
        \item Vorschlag für Akzeptanz-Testing mit Roland, wird angeschaut
        \item Schön wäre, wenn Deploy von Agents ohne Zertifikat erfolgt und sich jeder Agent bei API-Endpunkt ohne Cert-Check meldet, wo ein User manuell bestätigt, dass das Cert dieses Agents signiert wird
        \begin{enumerate}
            \item wird abgeklärt
        \end{enumerate}
        \item Deployment:
        \begin{enumerate}
            \item Rails/CC wahrscheinlich einfach geforkt/geklont.
            \item Agent am liebsten als Paket (siehe z.B. via launchpad) mit Abhängigkeiten direkt integriert
            \item Definitiv benötigt: Benutzer- und Installationsanleitung CC und A
        \end{enumerate}
    \end{itemize}


\end{document}