\documentclass[class=scrbook,crop=false]{standalone}

\usepackage{../../../style}
%------------------------------------------------------------------------------
% Glossary
%------------------------------------------------------------------------------

\newglossaryentry{test}{name=TestTest, description={ist ein Test-Eintrag},url={http://www.test.com/}}

%------------------------------------------------------------------------------
% Acronyms
%------------------------------------------------------------------------------

\newacronym{avt}{AVT}{Arbeiten-Verwaltungs-Tool der \acrshort{hsr}}

\newacronym{ba}{BA}{Bachelorarbeit}

\newacronym[url={http://hsr.ch/}]{hsr}{HSR}{Hochschule für Technik Rapperswil}

\newacronym{ci}{CI}{Continuous Integration}

\newacronym{upd89}{UPD89}{Update Nine.ch}

\newacronym{vm}{VM}{Virtuelle Maschine}

% names & full names
\newcommand{\ubo}{U.\ Bosshard\xspace}
\newcommand{\ubos}{Ueli Bosshard\xspace}

\newcommand{\pch}{P.\ Christen\xspace}
\newcommand{\pchr}{Philipp Christen\xspace}

\newcommand{\proff}{Prof.\ Farhad Mehta\xspace}
\newcommand{\prof}{Prof.\ F.\ Mehta\xspace}

% \mytable{cols}{content}{caption}{lbl}
\newcommand{\mytable}[4]{
	\begin{table}[H]
		\centering
		\begin{tabularx}{\textwidth}{#1}
			\toprule
			#2
			\bottomrule
		\end{tabularx}
		\caption{#3}
\label{tab:#4}
	\end{table}
}


% uses todonotes
\newcommand{\xxx}[1][]{
	\ifthenelse{\equal{#1}{}}{\todo[inline]{TODO}}{\todo[inline]{TODO:\ #1}}
}

% pretty url without http:// or https://
\newcommand{\purl}[1]{%
	\href{#1}{\StrBehind{#1}{://}}%
}
\newcommand{\shellcmd}[1]{\\\indent\indent\texttt{\footnotesize\# #1}\\}

\newcommand*{\zeroOfThree}{\FiveStar\FiveStar\FiveStarOpen}
\newcommand*{\oneOfThree}{\FiveStar\FiveStarOpen\FiveStarOpen}
\newcommand*{\twoOfThree}{\FiveStar\FiveStar\FiveStarOpen}
\newcommand*{\threeOfThree}{\FiveStar\FiveStar\FiveStar}

\newcommand*\cleartoleftpage{%
  \clearpage
  \ifodd\value{page}\hbox{}\newpage\fi
}


% Remove section numbering
\renewcommand*\thesection{}

% Remove spacing between section numbering and section titl
\makeatletter
\renewcommand*{\@seccntformat}[1]{\csname the#1\endcsname}
\makeatother

\begin{document}
	
	\section*{Projektsitzung 21. März 2016}
	
	\begin{tabular}{ll}
		\textbf{Datum} & 21.3.2016 \\
		\textbf{Zeit} & 13:10 \textendash 13:55 Uhr \\
		\textbf{Ort} & \acs{hsr} \\
		\textbf{Anwesende} & \proff \\ & \ubos \\ & \pchr
	\end{tabular}
	
	\subsection*{Traktanden}
	
	\begin{itemize}
		\item End of Elaboration
        \item Review Domainmodell und UseCases
	\end{itemize}
	
	\subsection*{Protokoll}

	\begin{itemize}
        \item Finale Aufgabenstellung unterzeichnet
        \item Use-Cases OK.
        \begin{itemize}
            \item Nicht zu technisch werden lassen, ohne implementationsspezifische Details (konkret: FQDN/Zertifikate entfernen). Aber: kann bei Brief Format als 'Technische Details' erwähnt werden.
            \item 'Read-Only-User' zu 'User' umbenennen
        \end{itemize}
        \item Domänenmodell
        \begin{itemize}
            \item Optionale Punkte, z.B. delay/API-Call/etc. --> entfernen und wenn es implementiert wurde, wieder einfügen
            \item Assoziationen wenn möglich bennen, z.B. User erstellt Job
            \item Anstatt State z.B. System Update-State verwenden
            \item Sprache einheitlich halten
            \item System: eventuell required restart auslagern in Notifications
            \item Zwischen-Klassen verwenden bei n-n-Beziehungen (Update/System, Package/PackageGroup, Package/System)
            \item State eventuell auslagern aus Task und SystemUpdate-Beziehung
            \item Vorschlag: Task-Execution von Task-Spezifikation trennen
            \item Permissions als eigene Klasse
            \item Das Domainmodell soll Konzepte beinhalten, nicht das Klassenmodell
        \end{itemize}
        \item Nächste Woche wegen Ostern kein Meeting. Bei Bedarf geht auch ein Email
    \end{itemize}
	
    \subsection*{Tasks bis zur nächsten Sitzung}
    
    \begin{enumerate}
        \item Domain-Model überarbeiten
    \end{enumerate}


\end{document}