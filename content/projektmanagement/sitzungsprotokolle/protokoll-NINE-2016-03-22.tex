\documentclass[class=scrbook,crop=false]{standalone}

\usepackage{../../../style}
%------------------------------------------------------------------------------
% Acronyms
%------------------------------------------------------------------------------

\newacronym{avt}{AVT}{Arbeiten-Verwaltungs-Tool der \acrshort{hsr}}

\newacronym{ba}{BA}{Bachelorarbeit}

\newacronym[url={http://hsr.ch/}]{hsr}{HSR}{Hochschule für Technik Rapperswil}

\newacronym{ci}{CI}{Continuous Integration}

\newacronym{upd89}{UPD89}{Update Nine.ch}

\newacronym{vm}{VM}{Virtuelle Maschine}

\newacronym{wsus}{WSUS}{Windows Server Update Services}
% names & full names
\newcommand{\ubo}{U.\ Bosshard\xspace}
\newcommand{\ubos}{Ueli Bosshard\xspace}

\newcommand{\pch}{P.\ Christen\xspace}
\newcommand{\pchr}{Philipp Christen\xspace}

\newcommand{\proff}{Prof.\ Farhad Mehta\xspace}
\newcommand{\prof}{Prof.\ F.\ Mehta\xspace}

% \mytable{cols}{content}{caption}{lbl}
\newcommand{\mytable}[4]{
	\begin{table}[H]
		\centering
		\begin{tabularx}{\textwidth}{#1}
			\toprule
			#2
			\bottomrule
		\end{tabularx}
		\caption{#3}
\label{tab:#4}
	\end{table}
}


% uses todonotes
\newcommand{\xxx}[1][]{
	\ifthenelse{\equal{#1}{}}{\todo[inline]{TODO}}{\todo[inline]{TODO:\ #1}}
}

% pretty url without http:// or https://
\newcommand{\purl}[1]{%
	\href{#1}{\StrBehind{#1}{://}}%
}
\newcommand{\shellcmd}[1]{\\\indent\indent\texttt{\footnotesize\# #1}\\}

\newcommand*{\zeroOfThree}{\FiveStar\FiveStar\FiveStarOpen}
\newcommand*{\oneOfThree}{\FiveStar\FiveStarOpen\FiveStarOpen}
\newcommand*{\twoOfThree}{\FiveStar\FiveStar\FiveStarOpen}
\newcommand*{\threeOfThree}{\FiveStar\FiveStar\FiveStar}

\newcommand*\cleartoleftpage{%
  \clearpage
  \ifodd\value{page}\hbox{}\newpage\fi
}


% Remove section numbering
\renewcommand*\thesection{}

% Remove spacing between section numbering and section titl
\makeatletter
\renewcommand*{\@seccntformat}[1]{\csname the#1\endcsname}
\makeatother

\begin{document}
	
    \section*{Meeting Nine.ch, 22. März 2016}
    
    \begin{tabular}{ll}
        \textbf{Datum} & 22.3.2016 \\
        \textbf{Zeit} & 16:00 \textendash 17:00 Uhr \\
        \textbf{Ort} & \acs{hsr} \\
        \textbf{Anwesende} & \sasie \\ & \rulrich \\ & \ubos \\ & \pchr
    \end{tabular}
    
    \subsection*{Traktanden}
    
    \begin{itemize}
        \item Ergebnisse von Elaboration-Phase
        \item Definition Sprint 1
        \item Workflow / Rules
    \end{itemize}

	\subsection*{Fragen}
	
	\begin{itemize}
        \item Rails 5? (Web-Sockets!) -> Ja, aber Achtung, abklären ob alle verwendeten GEMs Rails 5 kompatibel sind
        \item Test-Umgebung -> Bestellt, 10 VMs
    \end{itemize}
    
    \subsection*{Protokoll}
    
	\begin{itemize}
        \item Workflow
        \begin{itemize}
            \item Problem bei Failover-Paaren: Failover muss manuell getriggert werden. --> Serieller Ablauf funktioniert nicht wie angedacht
            \item Eventuell: Gruppe definieren, welche nicht gleichzeitig aktualisiert werden darf + warnen
            \item Generell: Workflow-Feature eher "nice-to-have".
            \item Resultat: Regelwerk wird weggelassen, Sam macht entsprechendes E-Mail
        \end{itemize}
        \item Domain-Model: OK, Permission-System "gute Idee, macht Sinn"
        \item Sprint 1: OK, "ambitioniert"
        \item Präsentation: Nine.ch würde gerne an Schlusspräsentation teilnehmen. Termin TBD
        \item Rails 5: OK, wenn Gems supported sind --> selbst definieren
        \item Testing-Hardware wird zur Verfügung gestellt, 4xUbuntu 12.04, 4xUbuntu 14.04, 1xCC, 1xRedmine. --> 10 Systeme
        \item Slack-Invite wird versendet
        \item Nächstes Meeting: 5.4.2016, 16:00
    \end{itemize}

\end{document}