\documentclass[class=scrbook,crop=false]{standalone}

\usepackage{../../../style}
%------------------------------------------------------------------------------
% Glossary
%------------------------------------------------------------------------------

\newglossaryentry{test}{name=TestTest, description={ist ein Test-Eintrag},url={http://www.test.com/}}

%------------------------------------------------------------------------------
% Acronyms
%------------------------------------------------------------------------------

\newacronym{avt}{AVT}{Arbeiten-Verwaltungs-Tool der \acrshort{hsr}}

\newacronym{ba}{BA}{Bachelorarbeit}

\newacronym[url={http://hsr.ch/}]{hsr}{HSR}{Hochschule für Technik Rapperswil}

\newacronym{ci}{CI}{Continuous Integration}

\newacronym{upd89}{UPD89}{Update Nine.ch}

\newacronym{vm}{VM}{Virtuelle Maschine}

% names & full names
\newcommand{\ubo}{U.\ Bosshard\xspace}
\newcommand{\ubos}{Ueli Bosshard\xspace}

\newcommand{\pch}{P.\ Christen\xspace}
\newcommand{\pchr}{Philipp Christen\xspace}

\newcommand{\proff}{Prof.\ Farhad Mehta\xspace}
\newcommand{\prof}{Prof.\ F.\ Mehta\xspace}

% \mytable{cols}{content}{caption}{lbl}
\newcommand{\mytable}[4]{
	\begin{table}[H]
		\centering
		\begin{tabularx}{\textwidth}{#1}
			\toprule
			#2
			\bottomrule
		\end{tabularx}
		\caption{#3}
\label{tab:#4}
	\end{table}
}


% uses todonotes
\newcommand{\xxx}[1][]{
	\ifthenelse{\equal{#1}{}}{\todo[inline]{TODO}}{\todo[inline]{TODO:\ #1}}
}

% pretty url without http:// or https://
\newcommand{\purl}[1]{%
	\href{#1}{\StrBehind{#1}{://}}%
}
\newcommand{\shellcmd}[1]{\\\indent\indent\texttt{\footnotesize\# #1}\\}

\newcommand*{\zeroOfThree}{\FiveStar\FiveStar\FiveStarOpen}
\newcommand*{\oneOfThree}{\FiveStar\FiveStarOpen\FiveStarOpen}
\newcommand*{\twoOfThree}{\FiveStar\FiveStar\FiveStarOpen}
\newcommand*{\threeOfThree}{\FiveStar\FiveStar\FiveStar}

\newcommand*\cleartoleftpage{%
  \clearpage
  \ifodd\value{page}\hbox{}\newpage\fi
}


% Remove section numbering
\renewcommand*\thesection{}

% Remove spacing between section numbering and section titl
\makeatletter
\renewcommand*{\@seccntformat}[1]{\csname the#1\endcsname}
\makeatother

\begin{document}
	
	\section*{Projektsitzung 7. März 2016}
	
	\begin{tabular}{ll}
		\textbf{Datum} & 7.3.2016 \\
		\textbf{Zeit} & 13:10 \textendash 13:55 Uhr \\
		\textbf{Ort} & \acs{hsr} \\
		\textbf{Anwesende} & \proff \\ & \ubos \\ & \pchr
	\end{tabular}
	
	\subsection*{Traktanden}
	
	\begin{itemize}
		\item Überarbeitete Aufgabenstellung
        \item Vorschlag Projektplan
        \item Risiken
	\end{itemize}
	
	\subsection*{Protokoll}
	
	\begin{itemize}
		\item Review Aufgabenstellung soweit fertig, Kunde muss OK geben, per Mail genügt oder mindestens im Protokoll festgehalten
		\begin{itemize}
            \item Kleine Anpassung: Alle Kriterien duchnummerieren
        \end{itemize}
        \item Projektplan: Meilensteine festlegen (mit nine)
        \item Risiko: Ruby-KnowHow kein Risiko, dafür Regeln umsetzen (Aufwand/Vorstellung des Kunden)
        \begin{itemize}
            \item Ebenfalls Risiko: Komplexes Rechtesystem
        \end{itemize}
        \item Qualitätsmassnahmen: Akzeptanzkriterien für Erledigung von Tickets? (siehe Projektplan QM)
        \item Testing: Definition TDD (Test Driven Developpment)?
        \item UseCase: gem. Larman: Nomen + Verb / CRUD kennzeichnen (CRUD in Use Case Diagrams)
        \item Projektname: ist upd89 neutral?
\item Herr Mehta wird noch einen Experten zum Gegenlesen der BA suchen. Fokus liegt in der Umsetzung, weniger im Projektmanagement. Falls die Studierenden selber Vorschläge haben, können sie diese bis nächste Woche mitteilen.
	\end{itemize}
	
    \subsection*{Tasks bis zur nächsten Sitzung}
    
    \begin{itemize}
        \item Besuch nine, Entscheide protokollieren!
        \item Überarbeitung Projektplan, Meilensteine und Risiken gem. Input Kunde
    \end{itemize}


\end{document}