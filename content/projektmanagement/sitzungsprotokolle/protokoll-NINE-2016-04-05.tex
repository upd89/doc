\documentclass[class=scrbook,crop=false]{standalone}

\usepackage{../../../style}
%------------------------------------------------------------------------------
% Glossary
%------------------------------------------------------------------------------

\newglossaryentry{test}{name=TestTest, description={ist ein Test-Eintrag},url={http://www.test.com/}}

%------------------------------------------------------------------------------
% Acronyms
%------------------------------------------------------------------------------

\newacronym{avt}{AVT}{Arbeiten-Verwaltungs-Tool der \acrshort{hsr}}

\newacronym{ba}{BA}{Bachelorarbeit}

\newacronym[url={http://hsr.ch/}]{hsr}{HSR}{Hochschule für Technik Rapperswil}

\newacronym{ci}{CI}{Continuous Integration}

\newacronym{upd89}{UPD89}{Update Nine.ch}

\newacronym{vm}{VM}{Virtuelle Maschine}

% names & full names
\newcommand{\ubo}{U.\ Bosshard\xspace}
\newcommand{\ubos}{Ueli Bosshard\xspace}

\newcommand{\pch}{P.\ Christen\xspace}
\newcommand{\pchr}{Philipp Christen\xspace}

\newcommand{\proff}{Prof.\ Farhad Mehta\xspace}
\newcommand{\prof}{Prof.\ F.\ Mehta\xspace}

% \mytable{cols}{content}{caption}{lbl}
\newcommand{\mytable}[4]{
	\begin{table}[H]
		\centering
		\begin{tabularx}{\textwidth}{#1}
			\toprule
			#2
			\bottomrule
		\end{tabularx}
		\caption{#3}
\label{tab:#4}
	\end{table}
}


% uses todonotes
\newcommand{\xxx}[1][]{
	\ifthenelse{\equal{#1}{}}{\todo[inline]{TODO}}{\todo[inline]{TODO:\ #1}}
}

% pretty url without http:// or https://
\newcommand{\purl}[1]{%
	\href{#1}{\StrBehind{#1}{://}}%
}
\newcommand{\shellcmd}[1]{\\\indent\indent\texttt{\footnotesize\# #1}\\}

\newcommand*{\zeroOfThree}{\FiveStar\FiveStar\FiveStarOpen}
\newcommand*{\oneOfThree}{\FiveStar\FiveStarOpen\FiveStarOpen}
\newcommand*{\twoOfThree}{\FiveStar\FiveStar\FiveStarOpen}
\newcommand*{\threeOfThree}{\FiveStar\FiveStar\FiveStar}

\newcommand*\cleartoleftpage{%
  \clearpage
  \ifodd\value{page}\hbox{}\newpage\fi
}


% Remove section numbering
\renewcommand*\thesection{}

% Remove spacing between section numbering and section titl
\makeatletter
\renewcommand*{\@seccntformat}[1]{\csname the#1\endcsname}
\makeatother

\begin{document}
	
    \section*{Meeting \gls{nine}, 5. April 2016}
    
    \begin{tabular}{ll}
        \textbf{Datum} & 5.4.2016 \\
        \textbf{Zeit} & 16:00 \textendash 17:00 Uhr \\
        \textbf{Ort} & \acs{hsr} \\
        \textbf{Anwesende} & \sasie \\ & \rulrich \\ & \ubos \\ & \pchr
    \end{tabular}
    
    \subsection*{Traktanden}
    
    \begin{itemize}
        \item Ergebnisse von Sprint 1
        \item Präsentation Prototyp
        \item Preview/Abnahme Sprint 2
    \end{itemize}
    
    \subsection*{Fragen}
    
	\begin{itemize}
        \item Multi-Threading mit Rails
        \item Mail betreffend Regelwerk
        \item Testumgebung von \gls{nine}
        \item Erfahrungen/Tipps zu Unicorn, ActiveAdmin, Netzke?
    \end{itemize}
    
    \subsection*{Protokoll}
    
    \begin{itemize}
        \item Test-Umgebung wurde aufgesetzt, Zugangsdaten erhalten
        \item Bestätigungs-Mail betreffend Regelwerk wird geschickt -> haben wir erhalten
        \item Alternative für ActiveAdmin oder Netzke: Administrate
        \item Rails Umgebung (Multi-Threading): \gls{nine} verwendet Apache und mod \textunderscore passenger -> kann auch so verwendet werden
        \item Agents kontaktieren
        \begin{itemize}
            \item Für die Umsetzung ist Annahme OK, dass die Agents direkt erreichbar sind
            \item Sicherheit muss gewährleistet sein (Zugriff auf Agent muss abgesichert sein)
            \item \gls{nine} verwendet selbst ein Message-Bus (RabbitMQ) -> wäre Luxuslösung
        \end{itemize}
        \item Mini-Code-Review der API:
        \begin{itemize}
            \item kleinere Methoden/Klassen schreiben
            \item Weniger Business-Logik in Controller, mehr in Models.
            \item Eventuell Service-Klassen (Service Pattern) einführen, welche die Logik übernehmen
        \end{itemize}
        \item Samuel wird voraussichtlich an Abschlusspräsentation teilnehmen
        \item Identifikation der Pakete: Name + Architektur + Distribution
        \item Nächstes Meeting: 19.4. 16:00 bei \gls{nine}
    \end{itemize}


\end{document}