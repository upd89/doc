\chapter{Analyse}

\section{Domäne}

\section{Konkurrenz} \label{sec:analysis:competition}

Hier werden bereits existierende Lösungen für ähnliche Problemstellungen vorgestellt und analysiert sowie mit der Lösung \gls{UPD89} verglichen.

\subsection{UPD89}


\subsection{Landscape}
https://landscape.canonical.com/
Landscape von Canonical ist ein proprietärer Web-Service für das Managen von Ubuntu-Systemen an einem zentralen Ort.
Es kann Software verwaltet und installiert werden, ebenso Security-Updates. Updates können auch automatisch installiert oder wieder entfernt werden.
Landscape unterstützt neben Servern auch Desktops.

Features:
Deployment
Software und Update Management inkl. Repository-Verwaltung
Monitoring, inkl. Prozess-Zugriff sowie div. Hardware-Abfragen
RBAC (Role-Based Access Control)

Kosten:
per se Kostenlos, benötigt aber eine Advantage-Lizenz. Diese kostet im Minimum 320 USD pro Server pro Jahr.
\xxx[Source: http://www.ubuntu.com/management/ubuntu-advantage]

\subsection{Ansible}
https://www.ansible.com/

Ansible ist eine Plattform zur Automatisierung von Tasks und zur Orchestrierung. Sie kombiniert Softwareverteilung, Kommando-Ausführung und Konfigurationsmanagement.Server werden via SSH angesprochen, es ist kein Agent erforderlich.
Ansible Tower die zentrale Web-basierte Kommandozentrale und ist Open-Source.
\xxx[Source https://de.wikipedia.org/wiki/Ansible]

Features:
Applikations-Deployment
Konfigurations-Management
User-definierbare Playbooks für Tasks
REST-Schnittstelle für Einbinden in andere Systeme

Kosten:
Im Enterprise-Paket kostet Ansible Tower 50 USD pro Server pro Jahr für > 1000 Server.
\xxx[Source: https://www.ansible.com/pricing]

\subsection{Vagrant}
https://de.wikipedia.org/wiki/Vagrant_%28Software%29
https://www.vagrantup.com/

\xxx[Wirklich relevant?]


\subsection{Spacewalk}
http://spacewalk.redhat.com/
Spacewalk ist eine Open-Source Systemmanagement-Lösung für Linux-Systeme. Es basiert auf einer Community, von welcher auch der Support kommt.
Es bietet unter anderem auch ein Update-Management über Systeme und Systemgruppen hinweg. Das Ganze funktioniert über eine zentrale Web-Applikation.
Spacewalk unterstützt Fedora, CentOS, SLE und Debian.

Features:
Inventar über Hardware und installierte Software
Installation von Software und Updates
Konfigurationen verteilen und managen
Content-Verteilung

Kosten:
Gratis
\xxx[Source: http://spacewalk.redhat.com/faq.html ]


\subsection{WSUS}

\gls{WSUS}
https://technet.microsoft.com/de-de/windowsserver/bb332157.aspx

Windows Server Update Services ist eine proprietäre Update-Software von Microsoft für zentralisiertes Managen von Patches für Microsoft-Server.

Features:
Verteilen der Updates, einmaliges Herunterladen
Client-Komponente, welche mit dem Manager kommuniziert und Status-Details liefert
Gruppieren von Systemen

Kosten:
Kostenlos
\xxx[Source: https://technet.microsoft.com/en-us/windowsserver/bb332157]
