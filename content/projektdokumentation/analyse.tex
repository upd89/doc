\chapter{Analyse}

\section{Domäne}

\section{Konkurrenz} \label{sec:analysis:competition}

Hier werden bereits existierende Lösungen für ähnliche Problemstellungen vorgestellt und analysiert sowie mit der Lösung \gls{UPD89} verglichen.

\subsection{UPD89}

\newcommand{\competitor}[7]{
	\begin{tabularx}{\linewidth}{lX}
		\toprule
		\textbf{Produkt} & #1\\
		\midrule
		\textbf{Hersteller} & #2\\
		\textbf{Beschreibung} & #3\\
		\textbf{Features} & #4\\
    \textbf{Lizenz} & #5\\
		\textbf{Kosten} & #6\\
    \textbf{URL} & #7\\
		\bottomrule
	\end{tabularx}
}

\subsection{Landscape}

\competitor{Landscape}
{Canonical}
{Landscape von Canonical ist ein proprietärer Web-Service für das Managen von Ubuntu-Systemen an einem zentralen Ort.

Es kann Software verwaltet und installiert werden, ebenso Security-Updates. Updates können auch automatisch installiert oder wieder entfernt werden.

Landscape unterstützt neben Servern auch Desktops.}
{\begin{itemize}
\item Deployment
\item Software und Update Management inkl. Repository-Verwaltung
\item Monitoring, inkl. Prozess-Zugriff sowie div. Hardware-Abfragen
\item RBAC (Role-Based Access Control)
\end{itemize}}
{Proprietär}
{Kostenlos, benötigt aber eine Advantage-Lizenz. Diese kostet im Minimum 320 USD pro Server pro Jahr.\footnote{Quelle: http://www.ubuntu.com/management/ubuntu-advantage}}
{https://landscape.canonical.com/}

\subsection{Ansible}

\competitor{Ansible}
{AnsibleWorks, Inc.}
{Ansible ist eine Plattform zur Automatisierung von Tasks und zur Orchestrierung. Sie kombiniert Softwareverteilung, Kommando-Ausführung und Konfigurationsmanagement.Server werden via SSH angesprochen, es ist kein Agent erforderlich.

Ansible Tower die zentrale Web-basierte Kommandozentrale und ist Open-Source.\footnote{Quelle: https://www.ansible.com/subscription-agreement}}

{\begin{itemize}
\item Applikations-Deployment
\item Konfigurations-Management
\item User-definierbare Playbooks für Tasks
\item REST-Schnittstelle für Einbinden in andere Systeme
\end{itemize}}
{GNU General Public License v3}
{Im Enterprise-Paket kostet Ansible Tower 50 USD pro Server pro Jahr für > 1000 Server.\footnote{Quelle: https://www.ansible.com/pricing}}
{https://www.ansible.com/}


\subsection{Spacewalk}

http://spacewalk.redhat.com/

Spacewalk ist eine Open-Source Systemmanagement-Lösung für Linux-Systeme. Es basiert auf einer Community, von welcher auch der Support kommt.

Es bietet unter anderem auch ein Update-Management über Systeme und Systemgruppen hinweg. Das Ganze funktioniert über eine zentrale Web-Applikation.

Spacewalk unterstützt Fedora, CentOS, SLE und Debian.

Features:

\begin{itemize}
\item Inventar über Hardware und installierte Software
\item Installation von Software und Updates
\item Konfigurationen verteilen und managen
\item Content-Verteilung
\end{itemize}

Lizenz:

\gls{gplv2}

Kosten:

Gratis

\xxx[Source: http://spacewalk.redhat.com/faq.html ]


\subsection{WSUS}

\gls{WSUS}
https://technet.microsoft.com/de-de/windowsserver/bb332157.aspx

Windows Server Update Services ist eine proprietäre Update-Software von Microsoft für zentralisiertes Managen von Patches für Microsoft-Server.

Features:

\begin{itemize}
\item Verteilen der Updates, einmaliges Herunterladen
\item Client-Komponente, welche mit dem Manager kommuniziert und Status-Details liefert
\item Gruppieren von Systemen
\end{itemize}

Kosten:
Kostenlos
\xxx[Source: https://technet.microsoft.com/en-us/windowsserver/bb332157]
