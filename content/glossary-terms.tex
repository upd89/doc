%------------------------------------------------------------------------------
% Glossary
%------------------------------------------------------------------------------

\newglossaryentry{test}{
    name={TestTest},
    description={ist ein Test-Eintrag},
    url={http://www.test.com/}
}

\newglossaryentry{agent}{
    name={Agent},
    description={Komponente, die dem Control Center meldet, welche Updates ausstehen sind und die Updates auf Befehl des CC durchführt}
}

\newglossaryentry{controlcenter}{
    name={Control Center},
    description={Komponente, die Reports anzeigt und die Steuerung der Updates ermöglicht}
}

\newglossaryentry{certificateauthority}{
    name={Certificate Authority},
    description={unterschreibt Zertifikate, damit festgestellt werden kann, welche Zertifikate bzw. Systeme einander vertrauen können}
}

\newglossaryentry{job}{
    name={Job},
    description={umfasst einen vom Benutzer definierten Auftrag zum Aktualisieren von mehreren Paketen auf mehreren Systemen. Pro betroffenem System enthält der Job einen Task.}
}

\newglossaryentry{task}{
    name={Task},
    description={wird auf ein spezifisches System geschickt und beinhaltet eine Liste von zu aktualisierenden Paketen auf diesem spezifischen System}
}

\newglossaryentry{apt}{
    name={apt},
    description={Kurz für 'Advanced Packaging Tool', ein Paket-Manager für Debian- und Ubuntu-Systeme},
    url={https://manpages.debian.org/cgi-bin/man.cgi?query=apt&sektion=8}
}

\newglossaryentry{mockup}{
    name={Mockup},
    description={Vereinfachter Entwurf eines komplexeren Produkts (z.B. einer Webseite), welches annähernd so aussieht wie das Endprodukt, aber noch keine Funktionalität bietet. So können wichtige Design-Entscheidungen bereits vor dem ersten Prototypen gemacht werden.}
}

\newglossaryentry{scss}{
    name={SCSS},
    description={'Sassy CSS', erweitertes CSS mit Unterstützung von Variabeln, Funktionen und weiteren Annehmlichkeiten. Scss-Dateien werden durch den SASS-Compiler in CSS-Dateien umgewandelt.},
    url={http://sass-lang.com/}
}

\newglossaryentry{ajax}{
    name={AJAX},
    description={'Asynchronous JavaScript and XML', eine Web-Technologie wo Inhalte asynchron nach dem initialen Laden der Seite über Javascript dynamisch nachgeladen wird}
}

\newglossaryentry{pagination}{
    name={Pagination},
    description={Das Aufteilen von einer langen Liste in mehrere Seiten, um sie übersichtlicher zu machen und die Ladedauer zu minimieren.}
}

\newglossaryentry{crud}{
    name={CRUD},
    description={'Create, Read, Update, Delete', grundlegende Operationen an einem Datensatz}
}

\newglossaryentry{daemon}{
    name={Daemon},
    description={Ein im Hintergrund ablaufender Prozess}
}

\newglossaryentry{nine}{
    name={nine.ch},
    description={Das Unternehmen Nine Internet Solutions AG},
    url={https://nine.ch}
}

\newglossaryentry{pip}{
    name={pip},
    description={Paket-Manager für Python},
    url={https://pypi.python.org/pypi/pip}
}

\newglossaryentry{rup}{
    name={Rational Unified Process},
    description={Ein Prozess um Software zu entwickeln},
    url={https://www.ibm.com/developerworks/rational/library/content/03July/1000/1251/1251_bestpractices_TP026B.pdf}
}