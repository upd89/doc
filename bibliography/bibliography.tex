\begin{comment}
Im Literaturverzeichnis sind alle verwendeten Quellen (Bücher, Publikationen,
URL, evtl. Auch Hinweise auf Gespräche und Information, welche über die
Computernetze beschafft wurde) aufgeführt. Typischerweise werden die
Quellenangaben nummeriert (z.B. [1], [2], ...) und in der Reihenfolge geordnet,
wie sie im Bericht vorkommen. Man kann die Quellen auch mit den
(abgekürzten) Namen der Autoren und dem Erscheinungsjahr bezeichnen (z.B.
[Schueli90], [Shannon49], ...), wobei man die Einträge dann alphabetisch
ordnet. Jede Referenz ist so anzugeben, dass sie möglichst einfach auffindbar
ist (evtl. inklusive Seitenzahl, Bibliothek Bestellnummer). Eine Referenz, welche
die allgemeine Grundlage für ein ganzes Kapitel bildet, wird im Titel bzw. in
einer Fussnote aufgeführt. Für Referenzen aus dem Internet soll ein
kommentierter URL angegeben werden.
Beispiele zu Quellenangaben:
[1] C.E.Shannon, “Communication in the Presence of Noise”, Proceedings IRE,
Vol. 37, January, 1949, pp. 10-21.
[2] Telefonat vom 22.5.92 mit Herrn XY, Firma, Adresse, Telefon-Nummer.
[3] Informationen zu Sicherheit von Breitband-Anschlüssen im Privatbereich der
Firma cnlab AG, http://www.cnlab.ch/en/documents.html, letzter Zugriff am
31.1.2003
\end{comment}

\printbibliography